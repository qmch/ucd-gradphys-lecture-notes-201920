Last time, we introduced the idea of density matrices, and we suggested that it might be more useful to discuss not state vectors $\ket{\psi}$ but generally state operators $\ketbra{\psi}{\psi}$, since a density matrix $\rho$, i.e. a positive, normalized, Hermitian operator, is the general state of a quantum system.

We explicitly showed that for a state vector in a composite system
\begin{equation}
    \ket{\text{EPR}} = \frac{1}{\sqrt{2}} (\ket{00} + \ket{11}) \in \cH_\text{qubit} \otimes \cH_\text{qubit},
\end{equation}
we could trace out the second qubit to get a reduced state $\rho_1$, namely
\begin{equation}
    \rho_1 = \frac{1}{2} (\ketbra{0}{0} + \ketbra{1}{1}).
\end{equation}
This state is mixed; it has equal probability of being in the $\ket{0}$ state or being in the $\ket{1}$ state. That is, the trace is the process of forgetting about qubit 2 (being agnostic to its state) and constructing a state in $\cH_1$ the first qubit state such that observables $A$ measured only on qubit 1 will be consistent with the composite observable $A\otimes \II$ measured on the two-qubit system. In particular, its entropy is
\begin{equation}
    S_1 = \ln 2.
\end{equation}

Let's pick $\ket{00}\in \cH_1 \otimes \cH_2$. If we trace out qubit 2, now we have
\begin{equation}
    \rho_1 = \Tr_2 (\ketbra{00}{00}) = \ketbra{0}{0},
\end{equation}
and we can also compute the von Neumann entropy of this reduced state. It is
\begin{equation}
    S_1 = -1 \ln 1= 0.
\end{equation}

This is different from the EPR state. The $\ket{\text{EPR}}$ state is a pure state of 2 qubits, but it has \emph{non-trivial} correlations between qubits 1 and 2. Moreover, these are quantum correlations, not classical correlations. The presence of quantum correlations is quantum entanglement. That is, the $\ket{\text{EPR}}$ state is a pure entangled state, whereas the state $\ket{00}$ is a pure unentangled (product) state.

Recall that a pure state corresponds to a state vector in Hilbert space. It is a density operator with a single nonzero eigenvalue in its spectral decomposition. Equivalently, $\rho^2=\rho$. We can also check that $S(\rho) = -\Tr(\rho \log \rho)=0$.

Conversely, mixed states are not pure. That is, they fail these properties: $S(\rho)\neq 0, \Tr(\rho^2) <1, \rho^2 \neq \rho$. Equivalently $\rho_{ij}=\bra{e_i}\rho \ket{e_j}$ has more than one non-zero eigenvalue.

\subsection*{Entanglement}
To discuss entanglement, we must work in a composite system. It is easiest to define entanglement by explaining what entanglement is not. A pure state of a composite system with no correlation between the components is called \term{separable}. Such states are tensor products of states in each component (i.e. the individual Hilbert space). They have the form
\begin{equation}
    \ket{\Psi} = \ket{\psi_1} \otimes \ket{\psi_2},\quad \ket{\psi_i} \in \cH_i.
\end{equation}
For instance,
\begin{equation}
    \ket{00} = \ket{0} \otimes \ket{0} \text{ is separable}.
\end{equation}
But
\begin{equation}
    \ket{00} + \ket{01} + \ket{10} +\ket{11} \text{ is also separable},
\end{equation}
since this state is just $(\ket{0}+\ket{1})\otimes (\ket{0}+\ket{1}).$
Separable states $\implies$ unentangled.

Notice that if we construct the density matrix for such a state and trace out one component, then the reduced state is pure. (This implies that the entropy of the subsystem is zero.)
\begin{equation}
    \Tr_2((\ket{\psi_1}\otimes \ket{\psi_2})(\bra{\psi_1}\otimes \bra{\psi_2}) = \ketbra{\psi_1}{\psi_1} \braket{\psi_2}{\psi_2} = \ketbra{\psi_1}{\psi_1}.
\end{equation}

Now pure \term{entangled states} are just pure states that are not separable. Such states are (a subset of) linear combinations of separable states. For instance, the EPR state is a combination of two individually separable states $\ket{00}$ and $\ket{11}$. Similarly there are other Bell (EPR) states like
\begin{equation}
    \frac{1}{\sqrt{2}}(\ket{01} \pm \ket{10}),\quad \frac{1}{\sqrt{2}}(\ket{00} \pm \ket{11}).
\end{equation}

Not all linear combinations of separable states are entangled. For instance,
\begin{equation}
    \ket{00} + \ket{01} = \ket{0} \otimes (\ket{0} + \ket{1})
\end{equation}
is separable. The real way to tell is by taking the partial trace. \emph{The partial trace of a separable state is a pure state. The partial trace of an entangled state is a mixed state.}

That is, if $\ket{\Psi} \in \cH_1 \otimes \cH_2$ is a (pure) entangled state, then
\begin{equation}
    \rho_1 = \Tr_2(\ketbra{\Psi}{\Psi}) \text{ and } \rho_2 = \Tr_1(\ketbra{\Psi}{\Psi})
\end{equation}
are mixed states.%
    \footnote{They are operators, so they formally belong to the space of endomorphisms on $\cH_1$ and $\cH_2$. In particular they are not in the Hilbert spaces!}

Note that the subsystem entropies are equal:
\begin{equation}
    S_1 = -\Tr(\rho_1 \log \rho_1) = S_2 = -\Tr(\rho_2 \log \rho_2).
\end{equation}
If after tracing out part of a state $\ket{\Psi}$ we find that the subsystem entropy $S_1=S_2=0$, then the original state was separable.

We might be interested in the question of whether a general mixed state can always be purified, i.e. whether it can be written as the partial trace of a bigger system. The answer is yes, but not uniquely. For
\begin{equation}
    \rho_1 = \frac{1}{2}(\ketbra{0}{0} + \ketbra{1}{1})
\end{equation}
on $\cH_1$ can be obtained from
\begin{equation}
    \frac{1}{\sqrt{2}}(\ket{00} + \ket{11})
\end{equation}
or from
\begin{equation}
    \frac{1}{\sqrt{2}}(\ket{0000} + \ket{1111})
\end{equation}
or even
\begin{equation}
    \frac{1}{\sqrt{2}}(e^{i\phi_1} \ket{00} + e^{i\phi_2} \ket{11})
\end{equation}
for any real $\phi_1,\phi_2$ we like.%
    \footnote{There is a general recipe for purification. For a state $\rho$ on the Hilbert space $\cH$, simply take a \term{maximally entangled state} $\ket{\Omega} = \sum_i \ket{i}\otimes\ket{i} \in \cH \otimes \cH$, where the $\ket{i}$s form a complete set of states for $\cH$. Then write $\ket{\Psi} = \sqrt{d}(\sqrt{\rho} \otimes \II)\ket{\Omega}$. You can check that such a state is pure by construction and has the right reduced state. Again, see my quantum info notes for the details.}
% Pure states can be entangled or unentangled (product) states. In product states, the entropy of the reduced state%
%     \footnote{Either one. It doesn't matter which subsystem you trace out because the subsystem entropies are equal, by a property known as the Schmidt decomposition.}
% is zero.

For mixed states, we can also describe entanglement. Let $\rho$ be a density operator acting on $\cH_1 \otimes \cH_2$. Such an operator is said to be separable if
\begin{equation}
    \rho = \sum_i p_i (\rho_i^{(1)} \otimes \rho_i^{(2)}), \quad \sum p_i = 0, 0\leq p_i \leq 1.
\end{equation}
If not, it is entangled. Equivalently, $\rho$ admits a block decomposition
\begin{equation}
    \rho = p_1 \begin{pmatrix}
    \boxed{\rho_1^{(1)}} &\\
    & \boxed{\rho_1^{(2)}}
    \end{pmatrix} + p_2 \begin{pmatrix}
    \boxed{\rho_2^{(1)}} &\\
    & \boxed{\rho_2^{(2)}}
    \end{pmatrix}.
\end{equation}
Such a state also has entropy, but it is classical (Shannon) entropy from classical correlations.

\subsection*{Thermal entropy as von Neumann entropy}
Let us pick some quantum system with a Hilbert space $\cH$ and Hamiltonian $\hat H$. For instance, imagine the harmonic oscillator. It has a density operator
\begin{equation}
    \hat \rho_\beta = \frac{e^{-\beta \hat H}}{Z(\beta)},
\end{equation}
where $Z$ is the partition function
\begin{equation}
    Z(\beta) = \sum_n e^{-\beta E_n}.
\end{equation}
Notice that the partition function (the normalizing factor) is the sum
\begin{align}
    \Tr_{\cH} (e^{-\beta \hat H}) &= \sum_n \bra{E_n} e^{-\beta \hat H} \ket{E_n}\\
        &= \sum_n e^{-\beta E_n}.
\end{align}
We take $\beta = 1/k_B T$. We may write the elements of $\rho$ in the energy eigenbasis:
\begin{equation}
    \rho_\beta = \frac{1}{Z(\beta) } \sum_n e^{-\beta E_n} \ketbra{E_n}{E_n},
\end{equation}
which has precisely the interpretation of a statistical thermodynamic distribution, with the correct Boltzmann factors $e^{-\beta E_n}$ weighting each state $\ketbra{E_n}{E_n}$. With the partition function, we can therefore compute the free energy
\begin{equation}
    F= -k_B T \ln Z(\beta)
\end{equation}
and moreover the entropy
\begin{equation}
    S= -\P{F}{T},
\end{equation}
from the relation $dE=TdS + \text{work}, F=E-TS \implies dF = -SdT +\text{work}$.%
    \footnote{Work here is usually $pdV$ in differential form, and the Helmholtz free energy is just a Legendre transformation of the energy.}
But the (classical) entropy of this distribution is 
\begin{equation}
    S_\text{thermal}=-\sum_n p_n \ln p_n,
\end{equation}
where $p_n$ is the Boltzmann probability
\begin{equation}
    p_n \equiv \frac{e^{-\beta E_n}}{Z(\beta)}.
\end{equation}
And if we compute the von Neumann entropy, we find precisely that
\begin{equation}
    S_\text{vN} = -\Tr(\rho \ln \rho) = -\sum p_n \ln p_n = S_\text{thermal}.
\end{equation}

Given $\hat \rho_\beta$ acting on $\cH_i$, consider
\begin{equation}
    \ket{\text{TFD}}\in \cH \otimes \cH
\end{equation}
the \term{thermofield double} state. There's a Hamiltonian on $\cH \otimes \cH$, which we can make from the Hamiltonians on the individual $\cH$s:
\begin{equation}
    \hat H_\text{TFD} = \hat H \otimes \II - \II \otimes \hat H.
\end{equation}
We may write out the matrix elements
\begin{equation}
    \ket{TFD} = \sum \frac{e^{-\beta E_n/2}}{\sqrt{Z(\beta)}} \ket{E_n} \otimes \ket{E_n}.
\end{equation}
This is a purification of $\rho_\beta$%
    \footnote{As I defined above. This is as close as we can get to a canonical purification.}
and this is a (maximally) entangled pure state in a bipartite system whose function is to reproduce all of thermodynamics in the reduced system.