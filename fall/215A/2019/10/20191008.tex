Today we will introduce the postulates of QM. They are as follows:
\begin{enumerate}
    \item[I.] The state of a system is described by a \emph{ray}%
        \footnote{That is, a vector defined up to normalization. Two vectors that differ only by normalization should be considered as representing the same physical state.}
    in Hilbert space.%
        \footnote{For our purposes, a Hilbert space is a complex vector space.}
    \item[II.] States evolve by unitary evolution,
    \begin{equation}
        \ket{\psi(t)}=U(t;t_0) \ket{\psi(t_0)},
    \end{equation}
    such that $U^\dagger U= \II$.%
        \footnote{Intuitively, so long as time evolution is unitary, states just evolve by complex rotations in Hilbert space. Norms and in particular probabilities are therefore left unchanged by unitary evolution.}
    \item[III.] Physical observables%
        \footnote{That is, observables of closed quantum systems where the observer is \emph{not part of the system}. In Mukund's words, we deal with a ``meta-observer theory'' where the observer can make measurements on a quantum system. There are attempts (cf. Everettian ``many-worlds'' QM) to describe the observer as part of the system, but this leads to lots of confusion.}
    correspond to expectation values of linear Hermitian operators which naturally act on states in Hilbert space. That is, for $\cA=\cA^\dagger$, we have the expectation value $\bra{\psi}\cA\ket{\psi'}$.
    \item[IV.] \term{The Born rule.} A measurement of a physical observable $\cA$ yields exactly one of its eigenvalues $\lambda$ with probability $p(\lambda)$ given by 
    \begin{align*}
        p(\lambda)&=\bra{\psi} \cP_\lambda(\cA)\ket{\psi}\\
            &= \braket{\psi}{a_\lambda} \braket{a_\lambda}{\psi}\\
            &= ||\braket{a_\lambda}{\psi}||^2
    \end{align*}
    where $\cP_\lambda(\cA)$ is the projection operator onto $\ket{a_\lambda}$, the eigenstate with eigenvalue $\lambda$, and we recall that $\cA$ has a spectral decomposition $\cA=\sum \lambda \ket{a_\lambda}\bra{a_\lambda}.$
\end{enumerate}
There are different interpretations of measurement in quantum mechanics. Pragmatically speaking, they should all give the same predictions so it doesn't really matter for computations%
    \footnote{But see J.S. Bell, \emph{Speakable and Unspeakable in Quantum Mechanics} for more on the philosophy of QM.}
and we will present one of the common interpretations, the \term{Copenhagen interpretation}. In the Copenhagen interpretation, measurement projects the state vector $\ket{\psi}$ onto one of the eigenstates of the operator being measured.%
    \footnote{The counterpoint to this is the Everettian picture. According to Copenhagen, a measurement is a lossy operation, where we lose information upon making a measurement. In many-worlds, the wavefunction instead branches upon making a measurement. If you like, we discover which branch of the wavefunction we were really on, since we are part of the quantum system.}
This is what we might call ``collapse of the wavefunction.'' Notice that
\begin{equation}
    \cA \ket{\psi} = \sum_\lambda \lambda \ket{a_\lambda} \braket{a_\lambda}{\psi},
\end{equation}
where the final inner product is the projection of the original state $\ket{\psi}$ onto the eigenstate $\bra{a_\lambda}$, and represents a probability amplitude (namely, its modulus squared represents the probability of measuring $\lambda$).

\subsection*{Evolution in Hilbert space}
By our postulate, time evolution is governed by unitary operators.
\begin{thm}[Wigner]
    Any mapping of a complex vector space onto itself that preserves the inner product is implemented either by
    \begin{enumerate}
        \item[i)] a unitary operator (linear transformation) or
        \item[ii)] an anti-unitary operator (anti-linear).
    \end{enumerate}
\end{thm}
That is, in case i) we have $U(\alpha\ket{\psi})= \alpha U\ket{\psi}, \alpha \in \CC$, whereas for an anti-linear operator in ii) we have $U(\alpha\ket{\psi})= \alpha^* U\ket{\psi}$. Most operators in physics are unitary apart from one famous example, which is time reversal.%
    \footnote{This has to do with the $i$ in the Schr\"odinger equation. Most anti-unitary operators one encounters in QM are related to time reversal in some way. Moreover the fact that the operator is anti-unitary suggests that in QM and even in QFT, there is a special direction for time.}

What Wigner's theorem guarantees us is that every symmetry of Hilbert space can be implemented by a unitary operator or an anti-unitary operator. The state evolution is therefore given by the (unitary) time translation operator, which is in turn generated by the Hamiltonian.%
    \footnote{Here I mean generated in the Lie algebra sense. That is, the Hamiltonian produces infinitesimal (differential) translations in time, and the exponential of an generator gives the full translation. We'll see this soon.}

Let us now revisit the notion of expectation values, considering the VEV (traditionally ``Vacuum Expectation Value'') of an operator:
\begin{defn}
    The \term{expectation value} of an operator $\cA$ in a state $\ket{\psi}$ is given by
    \begin{align}
        \avg{\cA}&= \bra{\psi} \cA \ket{\psi}\\
            &= \sum \lambda \braket{\psi}{a_\lambda} \braket{a_\lambda}{\psi}.
    \end{align}
\end{defn}
We may also talk about the uncertainty (basically the standard deviation) of an operator.
\begin{defn}
    The uncertainty of an operator $\cA$ is given in terms of expectation values:
    \begin{equation}
        \Delta \cA =\sqrt{\avg{(\cA-\avg{A}\II)^2}}.
    \end{equation}
\end{defn}
This is a quantum generalization of the standard deviation from ordinary statistics-- we want to know the root mean squared deviation from the mean value $\avg{A}$.

\subsection*{Single-particle QM}
In single-particle quantum mechanics, we're interested in studying a particle moving in 1 dimension under some external potential $V$. Classically, we would use an action principle (we'll revisit this in the path integral formulation later). That is, we write down a Lagrangian
\begin{equation}
    \cL = \frac{1}{2} m\dot x^2 - V(x)
\end{equation}
and we extremize the action $\int dt \cL(x,\dot x)$. However, for this part of the course we will instead use Hamiltonian language:
\begin{equation}
    \cH=p \dot x -\cL,\quad p = \frac{\delta \cL}{\delta \dot x},
\end{equation}
where the Hamiltonian is the Legendre transform of the Lagrangian, and thus if the potential depends only on $x$ then
\begin{equation}
    \cH = \frac{1}{2m} p^2 +V(x),
\end{equation}
in terms of the conjugate momentum $p=m\dot x$. Then we can write down Hamilton's equations, treating $x$ and $p$ as independent variables to find
\begin{equation}
    \dot x = \P{\cH}{p},\quad \dot p - \P{\cH}{x}.
\end{equation}
A particle's state is therefore represented by a point in phase space, i.e. the space of positions and momenta. For a particle in 1 dimension, the phase space is $\RR^2=(x,p)$, and in particular the phase space has an additional geometric structure. Namely, it is a \term{symplectic space}, meaning that that a Poisson bracket is defined on this space.

\begin{defn}
    Given two functions $f(x,p),g(x,p)$, the \term{Poisson bracket} of $f,g$ is simply
    \begin{equation}
        \set{f,g}=\P{f}{x}\P{g}{p} -\P{f}{p} \P{g}{x}.
    \end{equation}
\end{defn}
Hamilton's equations are nice in terms of Poisson brackets-- they just become
\begin{equation}
    \dot x = \set{x,\cH}, \quad \dot p =\set{p,\cH}.
\end{equation}
Note that $\frac{df}{dt} =\set{f,\cH}+\P{f}{t}$, in the case where $f$ may explicitly depend on time. Here, $t$ is time parametrizing trajectories in phase space.

\subsection*{Canonical quantization}
Now for quantum mechanics. The procedure of going from a classical system to a quantum system is called \term{canonical quantization}. This is done by taking phase space variables $x,p$ and uplfiting them to Hermitian operators $\hat x, \hat p$. We must also replace classical Poisson bracket relations between position and momentum with operator commutation relations (Lie brackets). Classically, the Poisson bracket between $x,p$ is
\begin{equation}
    \set{x,p}=1.
\end{equation}
In QM, this becomes
\begin{equation}
    [\hat x, \hat p] = i\hbar,
\end{equation}
with the commutator defined as
\begin{equation}
    [\hat A,\hat B] \equiv \hat A \hat B - \hat B \hat A.
\end{equation}

The commutator satisfies the following properties:
\begin{enumerate}
    \item[i)] linearity, $[\hat A,\alpha_1 \hat B + \alpha_2 \hat B_2] = \alpha_1 [\hat A,\hat B_1]+\alpha_2 [\hat A,\hat B_2]$ and similar for the first argument;
    \item[ii)] The Jacobi identity,
    \begin{equation}
        [\hat A,[\hat B,\hat C]] + [\hat B,[\hat C, \hat A]] + [\hat C,[\hat A, \hat B]]=0
    \end{equation}
    \item[iii)] Leibniz rule, $[\hat A,\hat B \hat C]= [\hat A,\hat B]\hat C + \hat B[\hat A,\hat C]$.
\end{enumerate}

The classical phase space is now replaced by a Hilbert space. All other observables become operators on Hilbert space, and this includes the Hamiltonian,
\begin{equation}
    H(x,p) \to \hat H (\hat x, \hat p).
\end{equation}
For a single particle in a simple potential, we can write it down immediately:
\begin{equation}
    \hat H(\hat x,\hat p) = \frac{1}{2m} \hat p^2 + V(\hat x).
\end{equation}

We should also note that classically, $x$ and $p$ commute. But quantum mechanically, they do not. Consider the classical function $xp$ on phase space. What is its quantum equivalent? Well, it could be $\hat x \hat p$. Or it could be $\hat p \hat x$. Or it could be the symmetrized sum $\frac{1}{2}(\hat x \hat p +\hat p + \hat x)$ (Weyl ordering). A priori, we do not know. This last option is popular, however.%
    \footnote{For another example of Weyl ordering, we would associate
    \begin{equation*}
        x^3 p \to \frac{1}{4}(\hat x^3 \hat p + \hat x^2 \hat p \hat x + \hat x \hat p \hat x^2 + \hat p \hat x^3).
    \end{equation*}
    }

Finally, let us define a wavefunction.
\begin{defn}
    A \term{wavefunction} is a state vector decomposed in the position eigenbasis. That is, let
    \begin{equation}
        \hat x\ket{x} = x\ket{x}
    \end{equation}
    indicate the position eigenbasis. Then
    \begin{equation}
        \psi(x)=\braket{x}{\psi},
    \end{equation}
    such that
    \begin{equation}
        \ket{\psi} = \int_{-\infty}^\infty dx\, \ket{x}\braket{x}{\psi} = \int_{-\infty}^\infty dx\, \ket{x} \psi(x).
    \end{equation}
\end{defn}