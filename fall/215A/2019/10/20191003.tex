\subsection*{Spectral theorem}
The spectral theorem roughly tells us that the eigenvectors of hermitian matrices will be guaranteed to form a good basis. Before we state the theorem formally, let us discuss the following. Given some Hermitian operator $\cZ$ with eigenspectrum
\begin{equation}
    \cZ\ket{z_n}=\zeta_n \ket{z_n}.
\end{equation}
Since we are guaranteed completeness%
    \footnote{We haven't shown it yet but suppose there's a vector that cannot be expressed as an eigenvector. We can reason to a contradiction. Probably in Shankar?}
it follows that
\begin{equation}
    \II\ket{v}=\ket{v} = \sum \ket{z_n} \braket{z_n}{v},
\end{equation}
so in fact reading this as an operator equation, since this is true for any $\ket{v}$,
\begin{equation}\label{resofidentity}
    \II = \sum_n \ket{z_n}\bra{z_n}.
\end{equation}
We call Eqn. \ref{resofidentity} the \term{resolution of the identity}, and this is true for any complete basis set.

That is, given $\ket{v}\in V$ and $\bra{w}\in V^*$, there is another operation, the \term{outer product} (denoted $\otimes$), which is a map from $(V,V^*)\to V\otimes V^*$. That is, it allows us to take a column vector and a row vector and combine them to form an operator (a matrix). Thus
\begin{equation}
    \ket{v}\bra{w} \in V\otimes V^*
\end{equation}
is an operator such that
\begin{equation}
    \paren{\ket{v}\bra{w}}\ket{z} = \braket{w}{z} \ket{v}.
\end{equation}

In tensor notation, we could write these as contravariant and covariant vectors as $V^\mu,\omega_\nu$ such that the outer product $(V\omega)^\mu{}_\nu$ has the correct indices.

We can also define the adjoint operation on operators in this notation,
\begin{equation}
    \paren{\ket{v}\bra{w}}^\dagger = \ket{w}\bra{v}.
\end{equation}
The adjoint operator may be thought of as a map $V\otimes V^* \to V^* \otimes V$.

Consider now
\begin{equation}
    \cP_n \equiv \ket{z_n}\bra{z_n}
\end{equation}
the projection operator onto $\ket{z_n}$. Naturally, $\cP_n \cP_n=\cP_n$ since
\begin{gather*}
    \cP_n \ket{v} = \ket{z_n} \braket{z_n}{v}\\
    \implies \cP_n^2 \ket{v} = \ket{z_n}\braket{z_n}{z_n}\braket{z_n}{v} = \cP_n \ket{v} \text{ if }\braket{z_n}{z_n}=1.
\end{gather*}
In this notation, we may equivalently write the identity as
\begin{equation}
    \II = \sum_n \cP_n.
\end{equation}
That is, the identity is the sum of all the projection operators. Moreover,
\begin{equation}
    \cZ = \sum_n \zeta_n \ket{z_n}\bra{z_n} = \sum_n \zeta_n \cP_n(\cZ),
\end{equation}
where the $\cP_n$ are projecting onto the eigenspectrum of $\cZ$.

Since this is given, it becomes easy to define functions of operators $f(\cZ)$ in terms of their eigenspectrum. Namely,
\begin{equation}
    f(\cZ)=\sum_n f(\zeta_n) \cP_n(\cZ)
\end{equation}
Note that an operator cannot have a nontrivial kernel (that is, it cannot have a zero eigenvalue) or else our completeness assumption fails. In addition, some functions have a finite radius of convergence and so the power series is not guaranteed to converge if the eigenvalues are unbounded.%
    \footnote{See Shankar 1.9.1.}

Let us also note that we've been working as though these spaces were finite-dimensional, but there are sometimes complications when we go to infinite dimensions. To every self-adjoint operator we can associate a 1-parameter family of projection operators $\cP(\lambda)$, parametrized by some $\lambda\in \RR$ such that $\cP(\lambda)$ satisfies
\begin{enumerate}
    \item[(i)] $\lambda_1 <\lambda_2 \implies \cP(\lambda_1) \cP(\lambda_2) = \cP(\lambda_2) \cP(\lambda_1) = \cP(\lambda_1)$. (That is, we always project onto smaller $\lambda$.)
    \item[(ii)] If $\epsilon>0$ then $\cP(\lambda+\epsilon)\ket{v} \to \cP(\lambda)\ket{v}$ as $\epsilon\to 0$. (In a sense this family varies continuously.)
    \item[(iii)] $\cP(\lambda)\ket{v}\to 0$ as $\lambda \to -\infty$.
    \item[(iv)] $\cP(\lambda)\ket{v}\to \ket{v}$ as $\lambda\to \infty$.
    \item[(v)] $\int_{-\infty}^\infty \lambda d\cP(\lambda)=\cZ$.
\end{enumerate}

\subsection*{Discrete spectrum}
Let's examine these assumptions in the discrete case.
\begin{equation}
    \cP(\lambda)= \sum_n \Theta(\lambda -\zeta_n) \cP_n (\cZ),
\end{equation}
where
\begin{equation}
    \Theta(x) = \begin{cases}
        1 & x>0\\
        0 & x\leq 0
    \end{cases}
\end{equation}
is the Heaviside step function. In the degenerate case we can project onto the corresponding subspace.

To put this in physics language, we can project onto energy eigenspaces. That is, $\lambda$ lets us take sums of the projection operators up to some energy eigenstate of our choice. As $\lambda\to -\infty$ we're looking at energies below the ground state, so there are no states to project on; as $\lambda \to + \infty$ we get all the energies and hence get back the original operator $\cZ$.

\subsection*{Continuous spectrum}
Let's consider a concrete example. For the position operator $X$, which we define as acting on a function $f(x)$ by
\begin{equation}
    Xf(x)=xf(x),
\end{equation}
we could try to solve the eigenvalue equation
\begin{equation}
    Xf(x) = \zeta f(x).
\end{equation}
We might be tempted to write the eigenfunctions as delta functions, $\delta(x-\zeta)$. But this would be a serious error, since the delta function is really a distribution in the space of linear functionals, and ought to live under integrals. That is,
\begin{equation}
    \int_{-\infty}^\infty dx \delta(x) f(x) =f(0)
\end{equation}
is the statement that the delta function should be thought of as only being meaningful when under an integral and paired with a function; it is more like a dual vector, since it takes a function and gives back a number.

Note that from the delta function we can also write other derivative expressions like
\begin{equation}
    \delta'(x):f(x)\to \RR,
\end{equation}
another functional such that
\begin{align*}
    \delta'[f(x)]&=\int_{-\infty}^\infty dx \, \delta'(x) f(x)\\
        &= -\int_{-\infty}^\infty dx\, \delta(x) f'(x) + (\text{boundary term}\to 0)\\
        &= -f'(0).
\end{align*}
Given appropriate smoothness of $f(x)$, anyway.

To generalize our one-parameter family of projection operators to the continuous case, we can write the projection operators for the position operator $X$ as
\begin{equation}
    \cP_X(\lambda) f(x) = \Theta(\lambda-x) f(x).
\end{equation}
That is, we may project onto $x$ up to $x=\lambda$. For notice that
\begin{equation}
    \bkt{\int_{-\infty}^\infty \lambda d\cP_X(\lambda)} f(x) = \int_{-\infty}^\infty \lambda \delta(\lambda -x) d\lambda \,f(x) = xf(x).
\end{equation}
Hence by (v) above we see that
\begin{equation}
    X=\int_{-\infty}^\infty \lambda d\cP_X(\lambda).
\end{equation}

\begin{thm}
    If $\cA,\cB$ are mutually commuting, self-adjoint operators, each with a complete set of eigenvalues, then $\exists$ a complete orthonormal set of eigenvectors for both $\cA$ and $\cB$ simultaneously.
\end{thm}

Since we are guaranteed this, the name of the game is to find the maximal set of commuting operators.%
    \footnote{Like we did with angular momentum, for instance, in finding $J^2$ and $J_z$.}
If we find a complete set of commuting operators, then we can not only simultaneously diagonalize all of them, but any other operator that commutes with the given set also has the same eigenbasis. 

In a sense, this is much like how we look for conserved quantities in classical mechanics to avoid solving second-order equations. We'll be particularly interested in operators that commute with the Hamiltonian, since energy eigenstates will coincide with eigenstates of those operators.