\subsection*{Quantum dynamics}
In a quantum theory, time evolution is given by a unitary operator,
\begin{equation}
    \ket{\psi(t)}=U(t,t_0)\ket{\psi(t_0)}.
\end{equation}
Time evolution is determined by the Hamiltonian, a hermitian operator $\hat H = \hat H^\dagger$, such that
\begin{equation}
    e^{i\hat H}
\end{equation}
is unitary.%
    \footnote{This is unitary because $(e^{i\hat H})^\dagger=e^{-i\hat H^\dagger}=e^{-i\hat H}=(e^{i\hat H})^{-1}.$ More generally it comes from the fact that Hermitian operators form a Lie algebra which generates unitary transformations. We can build finite transformations by exponentiating the appropriate infinitesimal transformation.}

For infinitesimal time evolution, notice that
\begin{equation}
    U(t_0+dt,t_0)=\II +\frac{dU}{dt}|_{t=t_0} dt + O(dt^2).
\end{equation}
If we choose $U$ such that
\begin{equation}
    \frac{dU}{dt}=-\frac{i\hat H}{\hbar},
\end{equation}
then we can recover
\begin{equation}\label{schrodingervector}
    i\hbar \P{}{t}\ket{\psi(t)}=\hat H \ket{\psi(t)},
\end{equation}
the \term{Schr\"odinger equation} in vector notation.

For the Hamiltonian
\begin{equation}
    \hat H = \frac{\hat p^2}{2m}+V(\hat x),
\end{equation}
we have
\begin{equation}\label{schrodingerwithpsquared}
    i\hbar \P{}{t}\ket{\psi(t)}= \bkt{\frac{\hat p^2}{2m}+V(\hat x)}\ket{\psi(t)}.
\end{equation}
We can switch to the position basis, writing
\begin{equation}
    \hat p = -i\hbar \P{}{x},
\end{equation}
from the canonical commutation relation $[\hat x,\hat p]=i\hbar$. If we then sandwich Eqn. \ref{schrodingerwithpsquared} with a $\bra{x}$, a position eigenstate, we get
\begin{equation}
    i\hbar \P{}{t} \psi(x,t) = -\frac{\hbar^2}{2m} \frac{\p^2}{\p x^2} \psi(x,t) + V(x) \psi(x,t),
\end{equation}
which is now the Schr\"odinger equation written in terms of the position space wavefunction.

The Schr\"odinger equation describes the evolution of $\ket{\psi(t)}$, but we can equivalently use it to understand $U(t,t_0)$. Notice that
\begin{equation}
    \P{}{t}U(t,t_0) = -\frac{i}{\hbar} \hat H(\hat x,\hat p,t) U(t,t_0)
\end{equation}
with $U(t_0,t_0)=\II$. This follows from Eqn. \ref{schrodingervector} by substituting in $\ket{\psi(t)}=U(t,t_0)\ket{\psi(t_0)}$ and noting the derivative is with respect to $t$, leaving $\ket{\psi(t_0)}$ fixed. We can therefore write a formal solution for this equation. It looks almost like an exponential, except for the fact that the operators in the Hamiltonian may not commute at different times.%
    \footnote{This leads to equal-time commutation relations in the appropriate picture of QM.}

Hence we get the formal solution for $U$ (sometimes called Dyson's formula), written in terms of \emph{time ordering}:
\begin{equation}
    U(t,t_0)=\cT \exp \bkt{-\frac{i}{\hbar} \int_{t_0}^t dt'\, \hat H(\hat x,\hat p,t')}.
\end{equation}
Here, the curly $\cT$ indicates time ordering, i.e. we must write the operators based on the time at which they are evaluated. We won't discuss this too much for now, since a proper discussion of this would basically be tantamount to explaining the path integral.

But the story becomes much simpler if the Hamiltonian is time-independent. In that case, our formal solution has no time ordering ambiguities, and we can immediately write down
\begin{equation}
    U(t,t_0)=\exp \bkt{-\frac{i}{\hbar}(t-t_0)\hat H(\hat x,\hat p)}.
\end{equation}
Moreover this makes it clear that energy eigenstates (i.e. eigenstates of the Hamiltonian) evolve in time by phases, $e^{-i E t/\hbar}$.

\subsection*{Schr\"odinger and Heisenberg pictures}
Until now, we have been working in the Schr\"odinger picture, where operators are left fixed, so that \emph{observables} carry no time dependence, and states are generically time-dependent. Thus expectation values are given by
\begin{equation}
    \bra{\psi(t)}\hat A \ket{\psi(t)},
\end{equation}
and all the time dependence is in our states' rotation in Hilbert space.

But there is an alternate way to do calculations, the \term{Heisenberg picture.} In the Heisenberg picture, state vectors carry no time dependence but operators do.%
    \footnote{If you like, we regroup all the time dependence into the operators.}
For a Schrodinger picture state $\ket{\psi(t)}$, it is certainly true that
\begin{equation}
    \hat A \ket{\psi(t)} = \hat A U(t,t_0) \ket{\psi(t_0)},
\end{equation}
where $t_0$ is some fixed earlier time. Since $U$ is unitary we can also write this as
\begin{equation}
    \hat A \ket{\psi(t)} = U(t,t_0) \paren{U^\dagger (t,t_0) \hat A U(t,t_0)}\ket{\psi(t_0)},
\end{equation}
so that%
    \footnote{Mukund mentioned some mnemonic for why daggers go on the left of operators. I don't know what his is, but this is how I remember it. Write the time-dependent expectation value in Schr\"odinger picture and regroup the $U$s into the operator.}
\begin{equation}
    \bra{\psi(t)}\hat A \ket{\psi(t)}=\bra{\psi(t_0)}\paren{U^\dagger (t,t_0) \hat A U(t,t_0)}\ket{\psi(t_0)}.
\end{equation}
Call $\ket{\psi(t_0)}$ the \term{Heisenberg state vector}, defined at some predetermined time $t_0$, and then the price we pay is that our operators have become time dependent,
\begin{equation}
    \hat A(t)=U^\dagger(t,t_0) \hat A(t_0)U(t,t_0),
\end{equation}
where $\hat A(t)$ indicates the Heisenberg picture version of $\hat A$.

Note that for a time-independent Hamiltonian,
\begin{equation}
    \hat A(t) = e^{\frac{i}{\hbar}(t-t_0)\hat H}\hat A(t_0) e^{-\frac{i}{\hbar}(t-t_0)\hat H},
\end{equation}
and the only way to evaluate this is to Taylor expand as
\begin{equation}
    \hat A(t) = \paren{\II + \frac{i}{\hbar}(t-t_0) \hat H+ \paren{\frac{i}{\hbar}}^2 (t-t_0)^2\frac{\hat H^2}{2!} + \ldots} \hat A(t_0) \paren{\II - \frac{i}{\hbar}(t-t_0) \hat H+ \paren{-\frac{i}{\hbar}}^2 (t-t_0)^2 \frac{\hat H^2}{2!} + \ldots}
\end{equation}
and we claim (exercise) that every other term is of the form of a nested commutator,
\begin{equation}
    \hat A(t_0)+ \frac{i}{\hbar}(t-t_0)[\hat H,\hat A(t_0)] + (\#)[\hat H,[\hat H,\hat A(t_0)]]+\ldots
\end{equation}
where $\#$ is some numerical factor we have not worked out yet. This fact will be important on the homework.

Incidentally, this tells us that where states evolve by rotations, operators generically transform in some complicated way. This has become a point of interest in the study of quantum chaos, how the time evolution of operators in the Heisenberg picture may depend very sensitively on the initial conditions.

\subsection*{Generalizations}
Here, we have considered the case of a single particle moving in 1D, but more generally we could have a particle moving in $d$ space direction so that the position and momentum operators have $d$ components,
\begin{equation}
    \hat x_i, \, i = 1,\ldots ,d,\qquad \hat p_j,\, j=1,\ldots,d,
\end{equation}
obeying the commutation relations
\begin{equation}
    [\hat x_i, \hat p_j]=i\hbar \delta_{ij},
\end{equation}
and in the Schrodinger equation, the $\hat p^2$ becomes the Laplacian in $d$ spatial coordinates,
\begin{equation}
    \nabla^2 = \sum_{i=1}^d \frac{\p^2}{\p x_i^2}.
\end{equation}

However, the generalization to multiple particles is more subtle. If our one-particle states live in Hilbert spaces $\cH_1,\cH_2,\ldots$ then the total Hilbert space is not just the direct sum $\oplus$ but the tensor product
\begin{equation}
    \cH=\cH_1 \otimes \cH_2.
\end{equation}
Hence if $\set{\ket{\psi_i}}_{i=1}^n$ generate $\cH_1$ and $\set{\ket{\chi_j}}_{j=1}^m$ generate $\cH_2$, then $\cH$ is generated by \emph{tensor products}
\begin{equation}
    \ket{\psi_i} \otimes \ket{\chi_j},
\end{equation}
and $\cH$ is $m\times n$-dimensional. It is this tensor product structure that is at the heart of the characteristic weirdness of quantum mechanics, namely the phenomenon known as \term{entanglement}.

\subsection*{The harmonic oscillator (Shankar Ch. 7)}
The quantum harmonic oscillator is a wonderful thing.%
    \footnote{Not official lecture content. Just my (correct) opinion.}
To begin, the classical harmonic oscillator is basically just a particle sitting in a parabolic (quadratic) potential. Its Lagrangian is
\begin{equation}
    L=\frac{1}{2}m\dot x^2 -\frac{1}{2} m\omega^2 x^2,
\end{equation}
with corresponding Hamiltonian
\begin{equation}
    H=\frac{p^2}{2m} +\frac{1}{2} m\omega^2 x^2.
\end{equation}

Canonical quantization tells us to promote $x$ and $p$ to operators obeying the canonical commutation relation
\begin{equation}
    [\hat x,\hat p]=i\hbar,
\end{equation}
so that
\begin{align}
    \hat H &=\frac{\hat p^2}{2m} +\frac{1}{2}m\omega^2 \hat x^2
    \hat H\\ &=-\frac{\hbar^2}{2m}\frac{d^2}{dx^2} +\frac{1}{2}m\omega^2 \hat x^2,
\end{align}
where we have rewritten $\hat p$ in the position basis.

Let us try to find the eigenspectrum of the Hamiltonian, i.e. the state vectors such that
\begin{equation}
    \hat H \ket{\psi(t)}=E\ket{\psi(t)}.
\end{equation}
The Schr\"odinger equation tells us that energy eigenstates evolve in a special way, namely by
\begin{equation}
    \ket{\psi(t)}= e^{-iEt/\hbar}\ket{\psi(0)}.
\end{equation}
%
Hence the Schrodinger equation written in terms of the position basis wavefunctions becomes
\begin{equation}
    \paren{-\frac{\hbar^2}{2m}\frac{d^2}{dx^2} +\frac{1}{2}m\omega^2 \hat x^2}\psi(x) = E\psi(x).
\end{equation}

This is a second-order ODE for $\psi$:
\begin{equation}
    \frac{d^2\psi}{dx^2}+\paren{\frac{2mE}{\hbar^2}-\frac{1}{2} \frac{m^2 \omega^2}{\hbar^2} x^2}\psi = 0.
\end{equation}
We can simplify by defining
\begin{gather}
    \frac{E}{\hbar \omega} = \epsilon,\\
    x= \sqrt{\frac{\hbar}{m\omega}}\xi.
\end{gather}
In these variables, our equation becomes%
    \footnote{This is a good strategy for physical problems in general. Defining dimensionless variables often gives us a sense of the important scales in the problem, e.g. energy is naturally measured in units of $\hbar \omega$. Sometimes we also recognize a differential equation as e.g. Hermite's equation or Legendre's equation when we write it in dimensionless form; a bit of familiarity with some common second-order equations can save a lot of time.}
\begin{equation}
    \frac{d^2\psi}{d\xi^2} +(2\epsilon-\xi^2)\psi=0.
\end{equation}
For such equations, we expect two families of solutions fitting the boundary conditions, one of which will be excluded on physical grounds. Typically we will have a condition on $\epsilon$, telling us that the energy is quantized.

Suppose we didn't have a computer to find the solutions to this equation right away. How do we study this equation? One way is to look at its limiting behavior. As $\xi\to \infty$, our equation reduces to
\begin{equation}
    \frac{d^2\psi}{d\xi^2}-\xi^2 \psi=0,
\end{equation}
with solutions
\begin{equation}
    \psi = c_1 \xi^m e^{\xi^2/2} + c_2 \xi^m e^{-\xi^2/2}.
\end{equation}
But this first term grows exponentially as $\xi\to \infty$ (equivalent to $x\to \infty$, so we should exclude that on physical grounds (normalization, if you like). 

In the limit $\xi\to 0$, we have instead
\begin{equation}
    \frac{d^2\psi}{d\xi^2} +2\epsilon \psi =0,
\end{equation}
with solutions
\begin{equation}
    \psi = a_1 \cos(\sqrt{2\epsilon}\xi) + a_2\sin(\sqrt{2\epsilon}\xi),
\end{equation}
so we haven't learned much from expanding about the origin.