%as they say in Pirates of the Caribbean, do we have the accord?
Today we proceed to the topic of partial differential equations (PDEs). These are the creatures we use to describe physical laws.

PDEs require us to know how to take derivatives on spaces of dimension greater than $1$. If our space $M$ where the function lives is sufficiently nice (it admits local coordinates) then we can define a directional derivative by defining a parametrized curve on $M$ and looking at how a function $f$ changes as we compare nearby points. That is, we can define
\begin{equation}
    \frac{\delta f}{\delta \vec l} = \vec l \cdot \grad f
\end{equation}
where
\begin{equation}
    \grad f = (\p_x f, \p_y f, \p_z f,\ldots)
\end{equation}
and the partial derivatives are given in the usual way by
\begin{equation}
    \p_x f(x,y,z) = \lim_{\Delta x \to 0} \frac{f(x+\Delta x,y, z) -f(x,y,z)}{\Delta x}.
\end{equation}

\begin{defn}
    A partial differential equation is an equation of the form
    \begin{equation}
        U(x_i,f,\p_{x_i}f, \p_{x_i} \p_{x_j} f, \ldots) =0.
    \end{equation}
    That is, our equation relates a function $f$, the coordinates $x_i$, and derivatives of $f$ with respect to the coordinates.
\end{defn}

\begin{defn}
    We shall call a PDE \term{linear} if upon replacing $f\to \lambda f$, our equation becomes
    \begin{equation}
        \lambda U + F(x),
    \end{equation}
    i.e. our equation just rescales up to an additive piece that does not depend on $\lambda$.
\end{defn}

Let us recall that we can always turn a single $n$th-order differential equation into a system of $n$ first-order equations by defining new variables (sort of like conjugate momenta) and rewriting. That is,
\begin{equation}
    \cL(D) X = 0 \leftrightarrow y_i' = f_i(x,y_i).
\end{equation}

We'll also restrict our interest to homogeneous linear first-order PDEs.%
    \footnote{Lit reference: Myskys on PDEs.}
That is, our equation takes the form
\begin{equation}
    \sum f_k(x) \p_k U = 0,
\end{equation}
where $\set{f_k(x)}$ is some set of $C^\infty$ smooth functions. A solution of this equation is a function $U=U(x)=C$ with some number of integration constants.
%the internets are crazy places.
That is, $U(x)$ maps points in the original space to a scalar value $C$. These are nothing more than equipotential surfaces in the original space. Sufficient data would then be a single point in the original space. Moreover, we can think about small variations in the ``potential'' value $C$. That is, $C\to C+\delta C$. Clearly, the function changes the most normal to the surface.

If we consider the tangent plane to the surface, we can also define the normal to the surface
\begin{equation}
    \vec n = \grad U.
\end{equation}
The tangent vectors on the surface are orthogonal to the normal,
\begin{equation}
    \vec f \cdot \vec n = 0,
\end{equation}
so that
\begin{equation}
    0=\vec f \cdot \grad U = \sum_k f_k \p_k U.
\end{equation}
This tells you that the $f_k(x)$ functions define a set of tangent vectors to a surface.%
    \footnote{Equivalently we're specifying a $n$-dimensional vector field on an $n$-dimensional space. We specify one constraint, the value of the function, so the surface is of codimension $1$.}
Most of the integration constants will fix the point we live at with respect to our surface. One constant will fix which surface we're on.%
    \footnote{Curiously, there's a link between the Frobenius theorem from differential geometry to the cosmological constant problem in effective field theory. See Weinberg '89, Reviews of Modern Physics}

\begin{exm}
    Consider the PDE
    \begin{equation}
        a\p_x \phi + b \p_y \phi =0.
    \end{equation}
    It is clear that we can translate between
    \begin{equation}
        \phi(x,y) =C
    \end{equation}
    and the parametric representation
    \begin{equation}
        x= x(C), y= y(C).
    \end{equation}
    If $a,b$ are constants then we are simply looking for the flows of a vector field which is constant everywhere. Its equipotentials are constant lines where $a\p_x + b\p_y$ is normal to the surface.
    
    The natural thing to do in the constant coefficient case is to change coordinates
    \begin{align}
        s&= ax +by\\
        t &= bx - ay.
    \end{align}
    We can rewrite our equation in terms of the new variables $s,t$ by the chain rule, i.e. because $\phi$ implicitly depends on $s$ and $t$ via
    \begin{equation}
        \phi(x,y) = \phi(x(s,t),y(s,t)),
    \end{equation}
    we can therefore write
    \begin{align}
        0 &= a\p_x \phi +b \p_y \phi\\
        &= a \paren{\P{\phi}{s} a + \P{\phi}{t} b} + b\paren{\P{\phi}{s} b -a \P{\phi}{t}}\\
            &= (a^2+b^2) \P{\phi}{s}.
    \end{align}
    Hence any function that depends only on $t$ is a solution,
    \begin{equation}
        \phi=f(t).
    \end{equation}
    Note that this worked since the new equation for $s$ was positive-definite. It can only become degenerate if $a^2+b^2=0$, in which case there's no equation.
\end{exm}
In other words, along curves of constant $t$, the function itself is constant since the solution is independent of $s$. Lines of constant $t$ are called \term{characteristics}. If you specify initial data on a line of constant $s$, then this sets initial data for the entire space. Conversely we are not free to specify arbitrary data on curves of constant $t$. In general, we must give initial data on a surface that crosses every characteristic at least once.

Another way to look at it is like this: suppose we have a parametrized curve $x=x_0(\lambda),y=y_0(\lambda)$, and we moreover are given initial data
$\phi(x_0(\lambda),y_0(\lambda))$ on this curve. Then we can write the full solution as
\begin{equation}
    \phi(x,y) = \phi(x_0,y_0) + \p_x \phi|_{x_0,y_0} \Delta x + \p_y \phi|_{x_0,y_0} \Delta y.
\end{equation}
That is,
\begin{equation}
    \frac{d\phi}{dl} = \p_x \phi \frac{dx}{dl} + \p_y \phi \frac{dy}{dl},
\end{equation}
by the chain rule, and we also know that
\begin{equation}
    a\p_x \phi + b \p_y \phi =0
\end{equation}
is our differential equation, so this provides us a system of equations for $\p_x \phi, \p_y \phi$. For this equation to have a solution, we must have
\begin{equation}
    \begin{vmatrix}
        \frac{dx}{dl} & \frac{dy}{dl}\\
        a& b
    \end{vmatrix} \neq 0.
\end{equation}
This tells us that
\begin{equation}
    b\frac{dx}{dl} - a\frac{dy}{dl} \neq 0,
\end{equation}
or equivalently our initial data must not be specified on a curve parallel to the characteristics. Otherwise, we are bound to run into pathologies.

Let us now consider something more exotic:
\begin{equation}
    a\p_x \phi + b \p_y \phi + q(x,y) \phi = F(x,y).
\end{equation}
If we replace $x$ and $y$ by $s,t$ as before, we find that
\begin{equation}
    (a^2 +b^2) \p_s \phi + \hat q(s,t) \phi = \hat F(s,t)
\end{equation}
where
\begin{equation}
    \hat q(s,t) = q(x(s,t),y(s,t))
\end{equation}
and $F$ is similar. Notice that this new equation has only one derivative, so now our equation has reduced to an ODE problem where $t$ just parametrizes which curve we are on.

Another natural generalization is to three variables,
\begin{equation}
    a\p_x \phi  + b\p_y \phi + c\p_z \phi = 0.
\end{equation}
If $a,b,c$ are constant we can certainly just define $s,t,u$ with say
\begin{equation}
    s=ax +by + cz
\end{equation}
and $t,u$ in the plane orthogonal to $s$. Then our equation simplifies as before, i.e.
\begin{equation}
    (a^2+b^2+c^2) \p_s \phi =0
\end{equation}
and $\phi= f(t,u) ={}$constant gives the solutions. Characteristics run along $s$ and take us off our initial data surface. We just follow the flow to extend our solution to the entire space.

In the 3D case, we now need initial data along two curves $l,l'$ in the surface, and then we can solve for $\p_x \phi,\p_y\phi,\p_z\phi$. The following determinant must be nonvanishing:
\begin{equation}
    \begin{vmatrix}
    \frac{dx}{dl} & \frac{dy}{dl} & \frac{dz}{dl}\\
    \frac{dx}{dl'} & \frac{dy}{dl'} & \frac{dz}{dl'}\\
    a & b & c
    \end{vmatrix}
\end{equation}

Finally, we will preview our first second-order equation,
\begin{equation}
    (\p_x^2 - \p_y^2) \phi =0.
\end{equation}
This is a wave equation. Clearly, this factorizes as
\begin{equation}
    (\p_x - \p_y)(\p_x+\p_y)\phi=0.
\end{equation}
If we now introduce new variables
\begin{align}
    u &= x+y\\
    v &= x-y,
\end{align}
then in terms of the new coordinates, we can rewrite this as
\begin{equation}
    \p_u \p_v \phi =0.
\end{equation}
These are none other than the null coordinates in Minkowski space. This tells us that functions of $u$ or $v$ alone are solutions to the equation, and so
\begin{equation}
    \phi(u,v)=f(u) + g(v)
\end{equation}
solve the wave equation, such that $f,g$ are at least twice differentiable.