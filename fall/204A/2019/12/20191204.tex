\begin{quote}
    \textit{``People could not live inside the shell of the earth. They would just float. Unless of course they lived on the equator. Real estate would be extremely precious.''}
    
    --Nemanja Kaloper
\end{quote}
Let's finish this. We were talking about separation of variables in spherical equations. We wanted to solve
\begin{equation}
    (\grad^2 + \vec k^2)\varphi = 0
\end{equation}
in terms of a solution which takes the form
\begin{equation}
    \varphi = R\Theta \Phi.
\end{equation}
Separation of variables gives us two angular equations:
\begin{equation}
    -\frac{1}{\Phi} \Phi''(\phi) = m^2,
\end{equation}
where we take $m$ to be positive and quantized in units of the period (for the regular periodic boundary conditions), and
\begin{equation}
    \frac{1}{\sin\theta} \frac{d}{d\theta} \paren{\sin\theta \frac{d\Theta}{d\theta}} - \frac{m^2 + \sin^2 \theta}{\sin^2\theta} \Theta = 0,
\end{equation}
which is just the Legendre equation in terms of a variable $x=\cos\theta$.

Finally, we have the radial equation
\begin{equation}
    \frac{1}{r^2} \frac{d}{dr} \paren{r^2 \frac{dR}{dr}} + k^2 R -\frac{\lambda R}{r^2} = 0.
\end{equation}

Solving the $\Phi$ equation readily gives us a phase 
\begin{equation}
    \Phi= e^{im \phi}
\end{equation}
The $\Theta$ equation gives us (associated) Legendre polynomials,
\begin{equation}
    \Theta=P_l^m(x=\cos\theta)
\end{equation}
which are derivatives of the standard Legendre polynomials $D_x^m P_l(x)$. Once we impose the quantization condition, the Legendre polynomials (and therefore their derivatives) are perfectly regular at the poles.%
    \footnote{We can see this from the Rodriguez formula for the Legendre polynomials.}
Finally, rewriting the radial equation yields
\begin{equation}
    r^2 R''+ 2r R' + (k^2 r^2 - \lambda)R=0,
\end{equation}
where
\begin{equation}
    \lambda = l(l+1).
\end{equation}
The solutions are Bessel functions $J_\nu$ for half-integer values of $\nu$.

Suppose we have a pure Laplace equation rather than Poisson. That is, the radial equation is
\begin{equation}
    r^2 R'' + 2rR' -l(l+1) R = 0.
\end{equation}
Then the solutions are just of the form $r^\alpha$: we see that
\begin{equation}
    \alpha(\alpha+1) = l(l+1),
\end{equation}
so that the solutions are $r^l,r^{-(l+1)}$. Hence our full solution to the Laplace equation in spherical coordinates is
\begin{equation}
    \varphi = \sum_{l,m}\paren{A_{lm}r^l +\frac{B_{lm}}{r^{l+1}}}P_l^m (\cos\theta) e^{im\phi}
\end{equation}
where $-l\leq m \leq l$ and $l=0,1,2,\ldots$. Of course, this is the multipole expansion. For $l=0$ the first term is constant ($r^0$) and the second term is $1/r$. When $l=0$ we must have $m=0$ and therefore the $l=m=0$ term gives us a constant plus the standard $1/r$ Coulombic potential.%
    \footnote{There's a cute property of associated Legendre polynomials, that $P_l^{-m} = (-1)^m P_l^m$.}

Look now at the $l=0,m=1$ term. The polar angular dependence is now $P_0^1(\cos\theta) = \cos\theta$ and the $1/r^{l+1}$ radial dependence becomes $1/r^2$. This gives us a $\cos\theta/r^2$ dependence, which we should recognize as the potential of an electric dipole. Higher $l$ will give us the quadrupole, octupole, and so on. The dominant behavior at large $r$ is the monopole behavior, unless the monopole moment vanishes.

However, notice that the first term $A_{lm}$ blows up at infinity, while the second term $B_{lm}$ blows up as $r\to 0$. We say the second describes an ``external solution'' outside some distribution of charges, while the first is an internal one.
%People could not live inside the shell of the earth. They would just float. Unless of course they lived on the equator. Real estate would be extremely precious.

More generally we might imagine patching together solutions based on some boundary conditions. Suppose for instance we impose a boundary condition on a sphere of some radius $a$, such that
\begin{equation}
    \vec E = \P{V}{\vec n} = -V_0 \cos\theta.
\end{equation}
These specify the normal derivative on a closed surface, so they are Neumann boundary conditions.
We may check that
\begin{equation}
    \int_0^\pi V_0 \cos\theta \sin\theta d\theta d\phi = \int_{1}^{-1} V_0 \cos\theta d(\cos\theta) d\phi =0.
\end{equation}
That is, there is no net charge on the sphere. This allows for the Gauss law to be satisfied; more generally, there are complications arising from such a calculation when these sorts of integrals do not vanish.

We now see by comparison to our general multipole expansion that $m=0$, since there must be no azimuthal (axial) angular dependence of our problem. We throw away the growing solution in the exterior region since it diverges as $r\to \infty$, and so our solutions are constrained to take the form
\begin{equation}
    \frac{B_{l0}}{r^{l+1}}P_l(\cos\theta).
\end{equation}
To match the other angular dependence with the boundary condition, we see that $l=1$, which means our solution is just
\begin{equation}
    \frac{B_{10}}{a^2} \cos\theta,
\end{equation}
on the boundary, and therefore
\begin{equation}
    V= - V_0 \paren{\frac{a}{r}}^2 \cos\theta.
\end{equation}
We recognize this as the dipole potential. Apparently the field of such a distribution is exactly the dipole potential outside!

What about the interior? We would keep the $r^l$ terms, and if we again keep only the $m=0,l=1$ terms, we now get
\begin{equation}
    V= -V_0 r\cos\theta = -V_0 z
\end{equation}
inside. The key element here is the orthonormality of the angular dependence eigenfunctions. Because these eigenfunctions are orthonormal and complete, we can expand a general solution in their basis and just match coefficients to the boundary conditions.

We can look at the Laplace equation in two dimensions,
\begin{equation}
    \Delta \Phi =0.
\end{equation}
Note that solutions $\Phi$ can have no extrema in the interior, for otherwise the Laplacian could not vanish-- suppose we chose axes such that $(\p_x^2 +\p_y^2)\Phi =0$. A maximum would have $\p_x^2 <0$ and $\p_y^2 <0$, which would imply that $\Delta \Phi >0$.

We shall now prove uniqueness of solutions. Suppose we had two solutions $\Phi_1,\Phi_2$ satisfying the boundary conditions. Notice that
\begin{equation}
    \Delta \phi = \Delta \Phi_1 - \Delta \Phi_2 =0,
\end{equation}
where
\begin{equation}
    \phi|_b =0
\end{equation}
vanishes on the boundary. Now notice that
\begin{align}
    0&=\int \phi \grad^2 \phi\\
        &= \int \grad(\phi \grad \phi) - (\grad \phi)^2\\
        &= \int_b d\vec A \cdot (\phi \grad \phi) - \int(\grad \phi)^2.
\end{align} 
Notice this quantity is like an energy, since $\grad^2 \phi \sim \rho$.
The first term vanishes since $\phi$ vanishes on the boundary. The second one is negative semi-definite, so this whole expression can only be zero if $\grad \phi=0$ identically. We conclude that $\phi$ is at most a constant, so these solutions differ at most by a constant; this is just our gauge symmetry. This argument works with Neumann boundary equations also; it's just the $\grad \phi$ term which vanishes on the boundary instead, which forces $\grad \phi$ to vanish everywhere.

For the wave equation, initial conditions are much simpler. Recall that solutions are a sum of a left-mover and a right-mover, i.e.
\begin{equation}
    \phi = f(x-t) + g(x+t),
\end{equation}
such that
\begin{align}
    \phi_0(x,0) = f(x) + g(x)\\
    \dot \phi_0(x,0) = -f'(x) + g'(x).
\end{align}
We can therefore integrate the second equation from $x-t$ to $x+t$ as
\begin{align}
    \int_{x-t}^{x+t} dx \, \dot \phi_0(x,0) &= -\int_{x-t}^{x+t} dx \, f'(x) + \int_{x-t}^{x+t} dx \, g'(x)\\
        &= -f(x+t) + f(x-t) + g(x+t) - g(x-t).
\end{align}
We can then evaluate the first equation at $x+t, x-t$ to get
\begin{equation}
    \phi(x+t,0) + \phi(x-t,0) = f(x+t) + f(x-t) + g(x+t) + g(x-t).
\end{equation}
Adding this with the lat equation gives us
\begin{equation}
    \phi_0(x+t,0) + \phi_0(x-t,0) + \int_{x-t}^{x+t} dx \, \dot \phi_0(x,0) = 2\bkt{f(x-t) + g(x+t)} = 2 \phi(x,t).
\end{equation}
Equivalently,
\begin{equation}
    \phi(x,t) = \frac{1}{2} \phi_0(x-t) + \frac{1}{2} \phi_0(x+t) + \frac{1}{2} \int_{x-t}^{x+t} dx \, \dot \phi_0(x).
\end{equation}
This solution is somewhat intuitive-- setting $t=0$ gives back $\phi_0(x)$, while taking the time derivative makes the first two terms cancel and the second one exactly match $\dot \phi_0$. The first two terms propagate the initial condition off to the left and right, while the integral adds up the influence of the initial ``velocities'' we specified at $t=0$.

Finally, we conclude with the heat equation,
\begin{equation}
    \p_t \phi = \p_x^2 \phi,
\end{equation}
which we will just treat in one spatial dimension. Let us again try separation of variables, $\Phi= TX$. Then
\begin{equation}
    \frac{T'(t)}{T} = \frac{X''(x)}{X}=\beta
\end{equation}
and each of these must be constant, where we have called the separation constant $\beta$. Then
\begin{equation}
    T = e^{\beta t}, \quad X = \bar Ae^{\sqrt{\beta} x} + \bar B e^{-\sqrt{\beta} x}.
\end{equation}
We can also define $\beta = -\alpha^2$. We ought to take $\beta \in \RR$ so that temperature does not oscillate in time. In fact, we should (on physical grounds) take $\beta <0$ in order to say our system is cooling down over time, and thus $\alpha$ is real and positive, which means that the $X$ dependence is just sines and cosines. That is,
\begin{equation}
    \Phi= \bkt{A\sin(\alpha x) + B \cos(\alpha x)} e^{-\alpha^2 t}
\end{equation}
or more generally
\begin{equation}
    \Phi= \int d\alpha C(\alpha) \bkt{A\sin(\alpha x) + B \cos(\alpha x)} e^{-\alpha^2 t}.
\end{equation}
%what's cooler than cold? Ice cold! What's cooler than ice cold? The cosmic microwave background, apparently.
We could therefore set initial conditions by specifying the heat distribution at $t=0$, and let it evolve. We might also have rewritten our solution in terms of a new dimensionless scale $u=\alpha x$, in which case we would see that the $e^{-\alpha^2 t}$ dependence becomes $e^{-t/u^2}$. If we suppose that $\Phi$ is therefore only a function of the coordinates in the combination
\begin{equation}
    U(x/\sqrt{t}),
\end{equation}
then
\begin{equation}
    \p_t \Phi = -\frac{1}{2} \Phi \frac{x}{t^{3/2}}.
\end{equation}
We arrive at an equation of the form
\begin{equation}
    2U'' + \zeta U'=0,
\end{equation}
which is first-order in $U'$. Hence the solution is
\begin{equation}
    U(\zeta) = C_1 \int_0^\zeta d\zeta' e^{-\zeta'{}^2/4} + C_2.
\end{equation}
The limit $\zeta\to \infty$ corresponds to $t\to 0$, and we can explicitly calculate
\begin{equation}
    U(\zeta \to \infty) = C_1 \int_0^\infty d\zeta' e^{-\zeta'{}^2/4} + C_2 = C_1 \sqrt{\pi} + C_2.
\end{equation}
Similarly we can evaluate $U(-\infty) = -C_1 \sqrt{\pi}$, which means that we can solve for $C_1,C_2$ in terms of initial conditions.

It follows that we can take derivatives with respect to $x,t$ of our original solution and generate new solutions. The solution is therefore
\begin{equation}
    \frac{1}{\sqrt{\pi}} \int_{-\infty}^\infty d\zeta e^{-\zeta^2} C (x-2\zeta \sqrt{t}).
\end{equation}
%7 problems on the final, 2 hours.