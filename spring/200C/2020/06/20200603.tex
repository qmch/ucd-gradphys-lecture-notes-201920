Today we'll talk about some practicalities of charged particle detection. In the days of Rutherford, scattered particles were detected by using a phosphor (luminescent) screen and a manual ``detector'' (a graduate student). These days, we use electronic detectors, but the principle is the same.

Charged particles passing through a medium mostly see the electron cloud. Atoms are about 1 Angstrom in scale, and the nucleus itself is on the order of femtometers. Electrons are bound to nuclei with about 1 eV (e.g. the ground state energy of the innermost electron in hydrogen is 13.7 eV). But charged particles passing through have much higher energies on the order of GeV.

It follows that charged particles can easily ionize materials; ionizing radiation is similar. In detectors, the electrons can be ionized and collected as a signal; we apply an overall electric field to prevent the electrons from recombining with the nuclei and instead drift them over to a collector where we can measure a signature.

Energetic charged particles have a characteristic lifetime based on their energy, and therefore an average distance traveled $c\tau$ before they decay. For a muon this is about $658$ meters, i.e. the product of the particle lifetime (in its rest frame) and the speed of light. In practice there is also a $\gamma$ boost from the fact these particles are moving relativistically, and can often travel much farther in our frame because of time dilation effects.
%``You are suffering radiation due to cosmic rays constantly.''

Consider a collision between an energetic heavy charged particle of mass $M$, energy $E=\gamma Mc^2$, momentum $p = \gamma \beta Mc$, with a free electron at rest. 