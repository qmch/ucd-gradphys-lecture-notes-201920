Today we'll discuss diffraction in various limits. We probably recall that $m\lambda=D\sin\theta$ gives the maxima in single slit diffraction, where $D$ is the distance from the single slit to the screen. In the double slit case, we get a similar pattern. Here, maxima occur at $m\lambda =d\sin\theta$, with $d$ the slit separation.

For circular apertures, we get a similar pattern but the first minimum occurs at $1.22\lambda = D\sin\theta$, where this value has to do with the first minimum of the Bessel function.

Long wavelengths correspond to a scattering limit; short wavelengths correspond to a diffraction limit ($\lambda/d \ll 1$). In scattering, we need to consider how the medium/target responds to the wave and re-radiates power; in diffraction, we just need to know how waves bend around obstacles.

In scalar diffraction theory, we get a wave equation
\begin{equation}
    \nabla^2 \vec A - \frac{1}{c^2}\frac{\p^2 \vec A}{\p t^2} =0.
\end{equation}
By taking the Fourier transform we get a Helmholtz equation
\begin{equation}
    \nabla^2 \tilde A +k^2 \tilde A=0
\end{equation}
and we can try to solve this with Green's function methods.

The Green's function for the 3D Helmholtz equation is (cf. Zangwill Eq.~10.79)
\begin{equation}
    G(\vec r', \vec r) = \frac{e^{\pm k|\vec r'-\vec r|}}{4\pi |\vec r'-\vec r|}
\end{equation}
which solves
\begin{equation}
    \nabla^2 G(\vec r',\vec r) + k^2 G(\vec r', \vec r) = -\delta(\vec r'-\vec r).
\end{equation}

Green's theorem(/identity) tells us we can write
\begin{equation}
    -a(\vec r)=\int_V\bkt{a(\vec r') \nabla'^2G(\vec r',\vec r)-G(\vec r', \vec r) \nabla'^2 a(\vec r')}dV' =\oint_S \bkt{a(\vec r') \nabla' G(\vec r',\vec r) - G(\vec r',\vec r) \nabla' a(\vec r')}dS'
\end{equation}
so we can rewrite
\begin{equation}
    a_k(\vec r) = -\int_{S_0} \bkt{a(\vec r') \nabla' G(\vec r',\vec r) - G(\vec r',\vec r) \nabla' a(\vec r')}dS'
\end{equation}
and now by specifying boundary conditions on the surface $S_0$ we can determine the solution $a$.

There are different limits we might take here, the near-field and far-field limits. Near-field is Fresnel diffraction, while far-field is Fraunhofer diffraction. Close to the aperture, edge effects are more prominent as the wave bends around the edges of the aperture; farther away, we recover the single-slit pattern.

Let's consider far-field diffraction, $\lambda \ll \vec r' \ll r$, where $\vec r'$ ranges over the aperture and $r$ is the location of the screen, which must be much larger than the size of the aperture. In this limit, we can write $\vec r'-\vec r \approx -\vec r$. Since we are working in a $\lambda \ll \vec r'$ limit, we also know that $k|\vec r'| \gg 1$ since $k=2\pi/\lambda$.

Now
\begin{equation}
    a_D(\vec r) = \frac{-i}{2\pi}\int_{S_0} \frac{k(\vec r'-\vec r) \cdot \uv n'}{|\vec r'-\vec r|}a(\vec r') e(?)
\end{equation}
For a circular aperture we have
\begin{equation}
    a_D(\vec r) = ia_0 \vec k \cdot \uv n' \frac{e^{ikr}}{2\pi r} \int_0^R \int_0^{2\pi} e^{-i\Delta \vec k \cdot \vec r'}d\phi' r'dr'
\end{equation}
where $e^{-i\Delta \vec k \cdot \vec r'}=e^{i\Delta k r' \cos\phi'}$, with $\Delta k$ indicating the difference between the incident wavevector and the diffracted wavevector,
\begin{equation}
    \Delta k = \vec k-\vec k_0.
\end{equation}
Performing the integral gives us a Bessel function,
\begin{equation}
    \int_0^{2\pi} e^{-i\Delta k \cos\phi'}d\phi' = 2\pi J_0(\Delta k r')
\end{equation}
and integrating the $r'$ integral gives
\begin{equation}
    \int_0^R J_0(\Delta k r') r' dr' = \frac{R}{\Delta k}J_1(\Delta k R).
\end{equation}
We find that
\begin{equation}
    a(\vec r) = ia_0 (\vec k \cdot \uv n') \frac{J_1(\Delta k R)}{\Delta kR} \frac{R^2 e^{ikr}}{r}.
\end{equation}
This is our amplitude function, and by squaring it to get the intensity we see that minima correspond to zeroes of the Bessel function.

Fresnel diffraction is the other limit, where $\lambda \ll \vec r \ll \vec r'$. Then we can expand
\begin{equation}
    |\vec r- \vec r'|^2 = r^2 + r'^2 -2\vec r \cdot \vec r'
\end{equation}
and
$k|\vec r'-\vec r|$ has an expansion in $r/r'$.
\begin{equation}
    kr = k\vec r' \cdot \uv r + \frac{k r'^2}{2r} - \frac{k(\vec r \cdot \uv r')^2}{2r}+\dots
\end{equation}

Assuming Dirichlet boundary conditions we can perform a Kirchhoff integral,
\begin{equation}
    a(x,y0 = -ia_0 \frac{kL}{2\pi} \int dx' dy' \frac{e^{ik|\vec r'-\vec r|}}{|\vec r'-\vec r|^2}
\end{equation}
and find the expression for Fresnel diffraction.