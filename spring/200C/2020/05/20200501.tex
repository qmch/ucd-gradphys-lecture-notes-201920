Today we'll continue with antennas. We can compute the time-average power from oscillating radiative sources. The power per unit surface area for an electric dipole is
\begin{equation}
    \frac{dP}{d\Omega} = \frac{p_0^2 \omega^4}{32\pi^2 \epsilon_0 c^3}\sin^2\theta,
\end{equation}
where $p(\vec t) = \vec p_0 e^{i\omega t}$ is the dipole moment as a function of time. If we perform the angular integral over $\int \sin^2\theta d\Omega$, we find that
\begin{equation}
    P_\mathrm{tot}^{\mathrm{ED}} = \frac{p^2 \omega^4}{12\pi \epsilon_0 c^3} = \frac{|\ddot p_0|^2}{6\pi \epsilon_0 c^3}.
\end{equation}
The reason we might want to write this as a time derivative is to make contact with expressions like the Larmor formula, which relate the acceleration to the power radiated.

One can write similar formulae for the magnetic dipole and the electric quadrupole.
\begin{equation}
    \frac{dP^\mathrm{(MD)}}{d\Omega} = \frac{\mu_0 m_0^2 \omega^4}{32\pi2 c^3}
\end{equation}
and
\begin{equation}
    P_\text{tot} = \frac{\mu_0 m_0^2 \omega^4}{12\pi c^3}.
\end{equation}
For the quadrupole,
\begin{equation}
    \frac{dP^\mathrm{(MD)}}{d\Omega} = \frac{\mu_0 \dddot Q_{zz}^2}{512 \pi^2 c^3} \cos^2\theta \sin^2\theta
\end{equation}
with the total power
\begin{equation}
    P = \frac{\mu_0 (3/2) \dddot Q_{zz}^2}{1420 \pi c^3}.
\end{equation}

For thin wire antennas, we can write the current as
\begin{equation}
    I(z,t) = I_0 \sin k(d-|z|) e^{i\omega t},
\end{equation}
a linear sinusoidal center-fed antenna. Then
\begin{equation}
   \vec j(\vec r,t) = I_0 \sin k(d-|z|)e^{i\omega t}\delta(x) \delta(y) \uv z. 
\end{equation}
We can then calculate a time-averaged power radiated by taking again the lowest-order nonvanishing term in our radiation approximation. We get
\begin{equation}
    \frac{dP}{d\Omega} = \frac{\mu_0 c I_0^2}{8\pi^2} \bkt{\frac{\cos(kd \cos \theta)-\cos(kd)}{\sin\theta}}^2.
\end{equation}
With $kd \gg 1$ (at wavelengths much smaller than the size of the antenna) this simplifies to
\begin{equation}
    \frac{dP}{d\Omega} =\frac{\mu_0 cI_0^2}{32\pi^2}(?)
\end{equation}

Fun facts about light and radiation.
\begin{itemize}
    \item The visible spectrum is defined by the frequencies of light we can see.
    \item X-ray is separated from ultraviolet by its capacity to ionize materials.
    \item Radio is separated from infrared by Joule heating (infrared warms things up)
    \item Gamma rays overlap with X-rays in their frequency band because they're distinguished by their sources (gamma from nuclear, X-rays from overall atomic/electronic emissions)
\end{itemize}

If we have a single charged particle moving as
\begin{equation}
    \vec j(\vec r,t) = q\vec v(t) \delta[\vec r- \vec r_0(t)]
\end{equation}
where $\vec v = \dot {\vec r}_0$, then we can calculate the radiation from the charge. The power per area is
\begin{equation}
    \frac{dP}{d\Omega} =\frac{\mu_0 q^2}{(?)\pi^2 c^3}\sin^2\theta.
\end{equation}

