Today we'll cover more realistic wave setups. We can write wave packets as integrals over Fourier modes,
\begin{equation}
    \vec E = \text{Re} \frac{1}{(2\pi)^3}\int d^3k \, \vec {\mathcal E}_\perp(\vec k) e^{-i(\omega t- \vec k \cdot \vec r)}
\end{equation}
and then the corresponding $B$-field is
\begin{equation}
    c\vec B = \text{Re} \frac{1}{(2\pi)^3}\int d^3k \, (\uv k \times \vec {\mathcal E}_\perp(\vec k)) e^{-i(\omega t- \vec k \cdot \vec r)},
\end{equation}
where $\vec k$ defines a wave vector (a direction of propagation).

Superposition and destructive interference allows us to constrain the transverse and longitudinal extent of the wavepacket. The $1/(2\pi)^3$ factor comes from taking the Fourier transform, working in radians; it is conventional. We're basically integrating over frequency components.

We can also write scalar wave packets in terms of the potentials rather than the fields. That is, we can write
\begin{equation}
    U(\vec r,t) = \frac{1}{(2\pi)^3}\int d^3k \, \hat U (\vec k) e^{-i(\omega(\vec k) t- \vec k \cdot \vec r)}
\end{equation}
in terms of an amplitude $u$.
%see notes

Suppose we take one dimension at a time,
\begin{equation}
    u(x,t=0;y,z) = \Int dk_x \uv u(k_x) e^{ik_x x}.
\end{equation}
We have to integrate over ``negative frequencies'' in order to treat waves as propagating forwards or backwards. We need to pick a distribution for $\uv u(k_x)$. Let's posit a Gaussian in $x$:
\begin{equation}
    \uv u(k_x) = \frac{1}{\sqrt{\pi}\Delta k_x} \exp \bkt{-(k_x-k_{0x})^2/\Delta k_x^2}.
\end{equation}
The characteristic spread in $x$ is $\sigma=\Delta k_x$, and the beam is centered on $k_{0x}$.
Why do we choose a Gaussian? Well, they are ubiquitous in nature and we have the mathematical apparatus to deal with them. If we insert this into our integral equation, we find that
\begin{equation}
    u(x,t=0) = \exp (ik_{0x} x) \exp (-x^2/(\Delta x)^2),
\end{equation}
so we get a beam which travels with wave number $k_{0x}$ and spatial width $\Delta x$ such that%
    \footnote{I think this is saturated for the Gaussian beam.}
\begin{equation}
    \Delta x \Delta k_x \geq 1/2.
\end{equation}

The time evolution describes how the packet moves in space. We can talk of a ``grup velocity,'' how fast the packet moves.
\begin{equation}
    v_g = \grad_k \omega(\vec k)|_{\vec k=\vec k_0} = \frac{d\omega(\vec k)}{d\vec k)}|_{\vec k=\vec k_0}.
\end{equation}
The frequency dependence defines a dispersion relations, so that e.g.
\begin{equation}
    \omega = vk = \frac{c}{n}k
\end{equation}
gives the group velocity as $c/n$, which depends on the index of refraction.
Then
\begin{equation}
    u(\vec r,t) = \frac{1}{(2\pi)^3} \int d^3 k \uv u(\vec k) e^{i(\vec k-\vec k_0) (\vec r - v_g t)} e^{-i(\omega_0t- \vec k_0 \cdot \vec r)}.
\end{equation}

Fourier transforms allow us to go between time and frequency domains (the amplitude $u(\vec r,t)$ and its Fourier transform $\hat u(\vec r, \omega)$).

If we substitute
\begin{equation}
    \hat u(\vec r, \omega) e^{-i\omega t}
\end{equation}
into the wave equation, we find that the wave equation becomes a condition on the Fourier transform,
\begin{equation}
    \nabla^2 u -\frac{1}{c^2} \frac{\p^2 u}{\p t^2} =0 \implies \paren{\nabla^2 + \frac{\omega^2}{c^2} }\hat u(\vec r,\omega)=0.
\end{equation}
We have transformed the wave equation into the \emph{Helmholtz equation}, which is also well-studied.

Now let us further make the paraxial approximation, i.e. we suppose the rays are close enough to the axis that $\sin\theta \approx \tan \theta \approx \theta.$ We start with the (source-free) Helmholtz equation,
\begin{equation}
    \nabla^2 \psi + k^2 \psi =0,
\end{equation}
where $\psi$ can be any component of $E$ or $B$, where $k=2\pi/\lambda$. We take
\begin{equation}
    \psi(\vec r,t) = \psi_0(\vec r,t) e^{i(kz-\omega t)},
\end{equation}
where $\psi_0$ is the beam profile. Taking the Laplacian gives us
\begin{equation}
    \nabla^2 \psi = \frac{\p^2 \psi_0}{\p x^2} +\frac{\p^2 \psi_0}{\p y^2} +\bkt{ik\P{\psi_0}{z} - k^2 \psi_0 + \frac{\p^2 \psi_0}{\p z^2}+ik\P{\psi_0}{z}}.
\end{equation}
Plugging back into the Helmholtz equation, the $k^2$ terms cancel. OUr paraxial apprixmation also says that the second derivative in $z$ is small, so we have
\begin{equation}
    \frac{\p^2 \psi_0}{\p x^2} +\frac{\p^2 \psi_0}{\p y^2} + 2ik \P{\psi_0}{z}=0,
\end{equation}
which we call the \term{paraxial wave equation}. In cylindrical coordinates we would instead have
\begin{equation}
    \frac{\p^2 \psi_0}{\p \rho^2} +\frac{1}{\rho} \P{\psi_)}{\rho} +2ik \P{\psi_0}{z}=0,
\end{equation}
where we can assume azimuthal symmetry.

The solution to this equation has the form
\begin{equation}
    \psi_0(\rho,z) = \frac{w_0}{w} \exp \bkt{-\frac{\rho^2}{w^2} - \frac{i\pi \rho^2}{\lambda R} + i \rho_0}
\end{equation}
where this beam has a gaussian width profle in $\rho$ and also oscillates in the radial direction, with
\begin{equation}
    w(z) = w_0\bkt{1+\paren{\frac{\lambda z}{\pi w_0^2}}^2}^{1/2}
\end{equation}
and
\begin{equation}
    R(z) = z\bkt{1+\paren{\frac{\pi w_0^2}{\lambda z}}^2},
\end{equation}
which defines a radius of curvature of the wavefront as a function of $z$. That is, at $z=0$ we have an ``infinite'' radius of curvature of the wavefront, and $w(z)$ defines a characteristic radius of the beam in the radial directions.

There's also an extent in $z$ where the beam remains reasonably confined:
\begin{equation}
    z_R = \frac{\pi \omega_0^2}{\lambda},
\end{equation}
which is the value of $z$ for which the beam area doubles ($w(z_R) = \sqrt{2} w_0$).

This is actually just the lowest order mode solution. For other symmetries we can expand in Hermite or Laguerre polynomials, as needed.%
    \footnote{We shouldn't be surprised at finding the Hermite or Laguerre polynomials, since the Helmholtz equation is closely related to the Schr\"odinger equation.}
    
In rectangular symmetry we get Hermite polynomials
\begin{equation}
    \psi_{0,mn}(z) = \frac{1}{w(z)} H_m\bkt{\frac{\sqrt{2}x}{w(z)}}H_n\bkt{\frac{\sqrt{2}y}{w(z)}} \dots
\end{equation}
(I didn't finish writing it down)
and for circular symmetry we would get Laguerre polynomials.

We can also discuss Fabry-Perot resonators, which are not covered in Zangwill. These are cavities which have two spherical mirrors set some distance from a center point. If we send light into the cavity (supposing the mirrors have some transmission), we can consider a symmetric resonator with equal focal lengths and find a nice expression for the waist of the beam. Moreover, we can work out what frequencies will constructively interfere in the cavity and find a characteristic frequency of the resonator.

There are a whole range of cavity setups we can construct based on the values of the focal lengths or equivalently the radii of curvature of the mirrors.