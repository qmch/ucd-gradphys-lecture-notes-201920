Today we'll discuss reflection and transmission of waves in simple matter (Zangwill 17.1-17.3).
\subsection*{Introduction}
As we've said, simple matter is matter that is linear, homogeneous, and isotropic. The polarization or magnetization responds linearly to applied fields, and the object is the same everywhere and in every direction.
We also require that our media is nonconductive, so there is no attenuation (compare a Faraday cage). In conductive media, the $E$ and $B$ fields can get out of phase. We should also restrict to nondispersive media, which responds to all frequencies in the same way (as opposed to e.g. a prism of acrylic, where the index of refraction depends on the frequency of light passing through).

\subsection*{Reflection and transmission}
Strictly, we should work with wave packets and Gaussian optics. However, for our discussion we'll make the simplifying assumption of dealing with plane waves. Suppose we have two media separated at $z=0$, defined by indices of reaction $n$ and $n'$. We'll send in a wave with unit wavevector $\uv k$ and two perpendicular directions $\uv p,\uv s$. Let's treat reflection in the $xz$-plane. We'll have $\uv p$ parallel to the plane of incidence, and $\uv s$ perpendicular.%
    \footnote{From \emph{senkrect}, perepndicular in German}

For $z<0, n = \sqrt{\mu \epsilon}$, while $z>0$ has an index of refraciton $n'= \sqrt{\mu' \epsilon'}$.

The incident electric field is
\begin{equation}
    \vec E_\text{inc} = \vec E_0 e^{i(\vec k \cdot \vec r - \omega t)},
\end{equation}
while the reflected field is
\begin{equation}
    \vec E_\text{ref} = \vec E_0'' e^{i(\vec k'' \cdot \vec r - \omega'' t)},
\end{equation}
and
\begin{equation}
    \vec E_\text{trans} = \vec E_0' e^{i(\vec k' \cdot \vec r - \omega't)}.
\end{equation}
Note that $z>0$ has just $\vec E_\text{trans}$, while $z<0$ has the sum $\vec E_\text{inc}+\vec E_\text{ref}$. At the interface $z=0$, we apply matching conditions and find that the fields must have the same space and time dependence,
\begin{equation}
    (\vec k \cdot \vec r - \omega t)=(\vec k' \cdot \vec r - \omega' t)=(\vec k'' \cdot \vec r - \omega'' t).
\end{equation}
This gives us some important constraints.
\begin{itemize}
    \item $\omega = \omega'=\omega''$ because the only time dependence is in $\omega t$
    \item $k_x=k_x'=k_x''$ because the variables in $\vec r$ are independent
    \item $k_y =k_y'=k_y''=0$
    \item $|k_z|=|k_z''|$ because $k=\omega/v = n\omega/c$
\end{itemize}
These give us the basic laws of optics:
\begin{itemize}
    \item Plane of incidence (if the wave starts in the $xz$ plane it stays in the plane)
    \item $\theta_\text{inc} = \theta_\text{ref}$
    \item Snell's law, $n\sin\theta_1 = n_2 \sin\theta_2$.
\end{itemize}

To get the transmission and reflection coefficients, we apply the boundary conditions.
\begin{itemize}
    \item $\Delta E_\text{tan}=0\implies (E_0+E_0'')_{x,y}=(E_0')_{x,y}$
    \item $\Delta D_\text{norm}=\sigma_f=0 \implies \epsilon(E_0 + E_0'')_z =\epsilon'(E_0')_z$
    \item $\Delta B_\text{norm}=0\implies (B_0+B_0'')_z = (B_0')_z$
    \item $\Delta H_\text{tan} = \vec K_f =0 \implies \frac{1}{\mu}(B_0+B_0'')_{x,y}= \frac{1}{\mu'}(B_0')_{x.y}$,
\end{itemize}
where tan is the tangential component and norm is the normal one.

Now we may choose a polarization. Let us define $\uv s=\uv y$ and $\uv p\equiv \cos \theta \uv x - \sin \theta \uv z$. For the $p$-polarization, we choose $\vec E$ to lie along $\uv p$ and $\vec B$ to lie along $\uv s$. We could have also chosen the $E$-field to lie along $\uv s$ instead, which is (quite reasonably) the $s$-polarization. We get four equations, though one is trivial. From the boundary conditions,
\begin{itemize}
    \item $(E_0+E_0'')\cos\theta=E_0'\cos\theta'$ ($\uv x$ component, the $\uv y$ equation is trivial)
    \item $\epsilon(E_0 + E_0'')\sin\theta =\epsilon'E_0'\sin\theta'$ (from the $\uv z$ matching condition on $D$)
    \item $0=0$ (from the $B_\text{norm}$ condition in this polarization)
    \item $\frac{1}{\mu}(\sqrt{\mu \epsilon}E_0+\sqrt{\mu\epsilon}E_0'')= \frac{1}{\mu'}\sqrt{\mu'\epsilon'} E_0'$ (from the $\uv y$ component of the $H_\text{tan}$ equation).
\end{itemize}

We're left with three independent equations and three unknowns: $\theta',E_0',E_0''$. %slide 4
Snell's law relates $\theta'$ to $\theta$ in terms of the indices of refraction, so we can rewrite the second equation above in terms of $\sin\theta$. 
Solving the system of equations gives us an expression for the transmitted wave,
\begin{equation}
    \frac{E_0'}{E_0} = \frac{2}{\frac{\mu n'}{\mu'n} + \frac{\cos\theta'}{\cos\theta}}\equiv t_{12,P}
\end{equation}
when the wave has the $p$-polarization. One can similarly compute the transmission and reflection coefficients for the other polarization. These are called the Fresnel coefficients, after Augustine-Jean Fresnel, who originally worked out these coefficients. Fresnel is also known for the design known as the Fresnel lens, which allows focusing of light without having a (spherical) lens that gets very thick in the middle.

Note that we can also define
\begin{equation}
    R=r_{12}^2,\quad T = \frac{\epsilon_2 v_2}{\epsilon_1v_1}t_{12}^2 \frac{\cos\theta_2}{\cos\theta_1}.
\end{equation}
There are critical angles we might consider, where a ray going from a slow to fast medium can experience total internal reflection.