Today we'll cover sections 16.1-16.4, on the wave equation.
%
Zangwill's discussion is of waves in vacuum, but in general we can do this with non-conducting, linear, homogeneous, isotropic media. We can break this down:
\begin{itemize}
    \item non-conducting: $\epsilon$ and $\mu$ are real (for conducting media, $\epsilon$ and $\mu$ have real and imaginary parts)
    \item linear: $\vec D = \epsilon \vec E, \vec B = \mu \vec H$ (there is a whole field of non-linear optics)
    \item homogenous: no spatial dependence (graded optics have many applications)
    \item isotropic: no preferred direction (anisotropic media also have interesting applications)
\end{itemize}
The fact that conducting media respond to cancel applied fields means that there is not just transmission but absorption; waves passing through are damped as the charge carriers in the medium reshuffle to counter the applied field.

Ferromagnets are an example of nonlinear media, as is the Kerr effect which changes the polarization of a wave. Nonlinearity usually kicks in when we hit physical limits on our ability to magnetize or polarize a substance, e.g. when the dipole moments in a ferromagnet are maximally aligned.

We sum this up as simple media, which vacuum certainly is, but in general there are many others that will satisfy these conditions. Consider a simple medium in the absence of sources, $\rho=0,\vec J=0$. The source-free Maxwell equations have a nice symmetry:
\begin{subequations}
    \begin{gather}
        \div \vec D = 0\\
        \div \vec B = 0\\
        \curl \vec E = -\P{\vec B}{t}\\
        \curl \vec H = \P{\vec D}{t}.
    \end{gather}
\end{subequations}

If we take the curl of Faraday's law, then
\begin{equation}
    \curl (\curl \vec E) = - \curl \paren{\P{\vec B}{t}}=-\P{}{t} (\curl \vec B).
\end{equation}
Now we use the double curl identity to rewrite the LHS as
\begin{equation}
    \curl (\curl \vec E) = \grad(\div \vec E) - \nabla^2 \vec E = -\nabla^2 \vec E
\end{equation}
by Gauss's law, and the RHS by Amp\`ere's law becomes
\begin{equation}
    -\P{}{t}\paren{\mu \epsilon \P{\vec E}{t}} = -\frac{1}{v^2}\frac{\p^2 \vec E}{\p t^2},
\end{equation}
which is a wave equation for each of the components of $\vec E$,
\begin{equation}
    \nabla^2 \vec E + \frac{1}{v^2}\frac{\p^2 \vec E}{\p t^2}=0
\end{equation}

Jackson takes a different approach; he writes the Maxwell equations and goes to the potential right away, with
\begin{equation}
    \vec E = -\grad \Phi - \P{\vec A}{t}, \vec B = \curl vec A.
\end{equation}
Inserting the potentials into Amp\`ere's law, we have
\begin{equation}
    \frac{1}{\mu} \curl(\curl \vec A) = \epsilon \P{}{t}\paren{-\grad \Phi - \frac{\p^2 \vec A}{\p t^2}}.
\end{equation}
Expanding out as before gives us
\begin{equation}
    \nabla^2 - \mu \epsilon \frac{\p^2 \vec A}{\p t^2} - \grad\paren{\div \vec A + \mu \epsilon \P{\Phi}{t}}=0.
\end{equation}
We can choose Coulomb gauge to write $\div \vec A =0$ and choose $\Phi=0$. we get Poisson's equation for $\Phi$ and the wave equation for $\vec A$, i.e.
\begin{equation}
    \nabla^2 \vec A - \mu \epsilon \frac{\p^2 \vec A}{\p t^2} =0
\end{equation}
and
\begin{equation}
    \epsilon\div(-\grad \Phi - \P{\vec A}{t})=0,
\end{equation}
so
\begin{equation}
    \nabla^2 \Phi=0.
\end{equation}

The wave equation was well-known by the time it was discovered Maxwell's equations had wave solutions. What was different was that EM waves could propagate through vacuum, without any medium necessary.

Solutions t the wave equations can be written in terms of null coordinates. If we define
\begin{equation}
    \xi = vt - \uv n \cdot \vec r = \hat n^\mu s_\mu,
\end{equation}
where $\uv n$ indicates the direction of propagation, then we can consider harmonic solutions:
\begin{equation}
    \vec  A( \xi) = \text{Re}\paren{\vec A_0 e^{-ik(vt - \uv n \cdot \vec r)}},
\end{equation}
such that
\begin{equation}
    kv = \omega, \vec k = \sqrt{\mu \epsilon \omega^2} \uv n,
\end{equation}
or equivalently
\begin{equation}
    \vec A(\vec r,t) = \text{Re}\paren{\vec A_0 e^{-i(\omega t - \vec k \cdot \vec r)}}
\end{equation}
The electric field is
\begin{equation}
    \vec E = -\P{\vec A}{t} = i\omega \vec A = i\omega \vec A_0 e^{-i(\omega t-\vec k \cdot \vec r)}.
\end{equation}
The magnetic field is
\begin{equation}
    \vec B = \curl \vec A = i\vec k \times \vec A = i\vec k \times \vec A_0 e^{-i(\omega t -\vec k \cdot \vec r)}.
\end{equation}
Note we could also pick the complementary null coordinate $\eta = vt +\uv n \cdot \vec r$, which simply propagates in the $-\uv n$ direction.

We could equivalently write the waves in terms of $e^{i\phi}$, where $\phi = \vec k \cdot \vec r - \omega t$, and this defines a phase velocity $v = \frac{d\vec r}{dt} = \frac{\omega}{k}$. The general solution need not be harmonic or periodic, but by a Fourier transform we can treat wave packets as a sum of Fourier modes.

From the Coulomb gauge condition $\div \vec A=0$, we can see that%
    \footnote{I think one can do this from a chain rule manipulation.}
\begin{equation}
    \uv n \cdot \frac{d\vec A}{d\xi}=0,
\end{equation}
so $\vec A \perp \uv n$ (up to a constant). It follows that if we choose $\Phi=0$, then
\begin{equation}
    \vec E = -\P{\vec A}{t} = -v \P{vec A}{\xi},
\end{equation}
while
\begin{equation}
    \vec B = \curl \vec A = -\uv n \times \P{\vec A}{\xi} = \frac{1}{v} \uv n \times \vec E \implies |\vec B| = \frac{1}{v} |\vec E|.
\end{equation}
We see that $\uv n, \vec E,$ and $\vec B$ form a right-handed orthogonal set, whre $\vec E$ and $\vec B$ are perpendicular and in phase.

The energy density is
\begin{equation}
    u=\frac{1}{2} \vec E \cdot \vec D + \frac{1}{2} \vec B \cdot \vec H = \frac{\epsilon}{2} E^2 + \frac{1}{2\mu} B^2.
\end{equation}
One can similarly analyze the Poynting vector and the angular momentum in the fields.

A final approach to this is to simply assume the form of a harmonic solution to the Maxwell equations (in particular, a plane wave) and write down a wave equation.

If we study plane waves, we have solutions with $1/v=\sqrt{\mu\epsilon}$ and a wavevector $k=\sqrt{\mu \epsilon}\omega$, with a phase velocity $\omega/k = 1/\sqrt{\mu \epsilon} =c/n$, where $n$ is the index of refraction. The index of refraction may be complex, which indicates absorption, and moreover it may be frequency-dependent.

We can superpose waves (e.g. plane waves) and in geneal construct
\begin{equation}
    \vec E = (\tilde E_1 \uv e_1 + \tilde E_2 \uv e_2) e^{i(\omega t + \vec k \cdot \vec r)},
\end{equation}
where $\uv n,\uv e_1$ and $\uv e_2$  form a right-handed set and $\tilde E_1,\tilde E_2$ are potnetially complex. That is, $\tilde E_{1/2} = E_{1/2} e^{i\phi_{1/2}}$.

If $E_1$ and $E_2$ have the same phase, we get linearly polarized light which oscillates in the plane. If $\phi_1 \neq \phi_2$, we get elliptic polarized light, where the $E$-field rotates around the axis of propagation. In particular when $|E_1|=|E_2|$ and $\phi_1 =\phi_2 +\pi/2$, we get \term{circularly polarized light}. We can consider special basis vectors
\begin{equation}
    \uv e_\pm = \frac{1}{\sqrt{2}}(\uv e_1 \pm i \uv e_2)
\end{equation}
where the signs indicate the helicity (whether the field spirals in a right-handed or left-handed way. $+$ is counterclockwise, $-$ is clockwise.