Today we'll continue our discussion of waveguides. As we said last time, one can propagate TM and TE waves down a conducting wave guide, but for TEM waves we need transmission lines-- the latter is only allowed in co-axial setups. In TE waves, the magnetic field points a little bit along the axis of propagation ($H_z \neq 0$); in TM waves, it is instead the electric field that points a bit along the $z$-axis.

In the Tm case, $E_z|_S=0$ on the boundary will give us sines, and in the TE case the derivative of $H_z$ being zero will give us cosines.

In general, $\vec E_\perp, \vec H_\perp,$ and $\uv z$ form a RH orthoganl set, with
\begin{equation}
    |\vec E_\perp| = Z |\vec H_\perp|.
\end{equation}
$E_\perp$ and $H_\perp$ have no phase difference, but either $E_\perp$ and $E_z$ or $H_\perp$ and $H_z$ are 90${}^\circ$ out of phase.

If we define $\gamma^2 = k_0^2 -k^2 =\mu \epsilon \omega^2$, solving for $k^2$ gives
\begin{equation}
    k_\lambda^2 \equiv \mu \epsilon \omega^2 - \gamma_\lambda^2,
\end{equation}
where we take $\omega$ to be given and $\gamma$ determined. Now this leads us to write a cutoff frequency
\begin{equation}
    \omega_\lambda = \frac{\gamma_\lambda}{\sqrt{\mu\epsilon}}
\end{equation}
such that above $\omega_\lambda$, the wavenumber
\begin{equation}
    k_\lambda = \sqrt{\mu \epsilon}\sqrt{\omega^2 -\omega_\lambda^2}
\end{equation}
becomes imaginary. That is, $\omega>\omega_\lambda$ gives $k_\lambda$ real and propagating waves, whereas for $\omega <\omega_\lambda$ we have cut-off modes and decaying exponential solutions.

In general we want to design our waveguide so that only one mode propagates. Consider a specific geometry, a rectangular waveguide. It has width $a$ in the $x$-direction and height $b$ in the $y$ direction. It's standard to take $a=2b$. Well, by separation of variables we can write down the solutions to the Helmholtz equation.
\begin{align}
    \psi_{nm} = E_0 \sin\paren{\frac{n\pi x}{a}}\sin\paren{\frac{m\pi y}{b}},n,m=1,2,3,\dots\quad \text{TM waves}\\
    \psi_{nm} = H_0 \cos\paren{\frac{n\pi x}{a}}\cos\paren{\frac{m\pi y}{b}},n,m=0,1,2,\dots\quad \text{TE waves}
\end{align}
Note that our convention for the lower indices is switched from Zangwill. The $\psi$s here are proportional to $E_z^{nm}$ and $H_z^{nm}$ respectively.

In either case, we can calculate the corresponding $\gamma$s:
\begin{equation}
    \gamma_{nm}^2 = \paren{\frac{n\pi}{a}}^2+\paren{\frac{m\pi}{b}}^2.
\end{equation}
THese are the eigenvalues, with
\begin{equation}
    \omega_{nm}=\frac{1}{\sqrt{\mu\epsilon}}\gamma_{nm}.
\end{equation}
We can see that the cutoff wavelength is $\lambda=2\pi/\gamma_{nm}$.

In the standard geometry, $a=2b$, we have
\begin{equation}
    \gamma_{nm} = \frac{\pi}{a} \sqrt{n^2+4m^2},
\end{equation}
which gives us the lowest modes as 
\begin{equation}
    \gamma_{10} = \pi/a, \gamma_{01}=\gamma-{20} = 2\pi/a.
\end{equation}
These must be TE modes, since one of their indices is zero. THe next ones up are
\begin{equation}
    \gamma_{11} = \sqrt{5}\pi/a, \gamma_{21} \sqrt{8} \pi/a,
\end{equation}
which can be TE or TM.

The corresponding cutoff frequencies are
\begin{equation}
    \omega_{nm} =\frac{c}{n} \gamma_{nm}, \quad k_{nm} = \frac{c}{n}\sqrt{\omega^2-\omega-{nm}^2}.
\end{equation}
We notice that the usual situation gives us
\begin{equation}
    \omega_{10} < \omega < \omega_{01}\text{ or } \omega_{20},
\end{equation}
then we can maximize the bandwidth for a desired frequency range.

The math is easiest for rectangular waveguides, but we can understand circular pipes as well; these give us Bessel functions in the radial direction instead.

Transmission lines with a coaxial geometry allow us to set up a potential difference between the two conductors, so TEM waves can propagate. Consider a coaxial cable with inner radius $a$ and outer raduis $b$. We can set up a solution for the potential
\begin{equation}
    \Phi = -\frac{\lambda}{2\pi \epsilon} \ln s,
\end{equation}
then
\begin{equation}
    E_s = -\P{\Phi}{s}
\end{equation}
and
\begin{equation}
    \vec E=(A/s) \uv s, \vec B=\frac{A}{cs}\hat{\gv \phi}.
\end{equation}

Note that dielectric waveguides use total internal reflection instead of metallic reflectivity to guide the wave. If the core has a higher index of refraction, $n_1 > n_2$, then total internal reflection is possible. The index of refraction can either change by step (two separate materials) or in a continuous, graded way. It depends on what frequency response we want from our fiber optic.

In closed resonant cavities rather than waveguides, we again solve Laplace's equation and fit boundary conditions. We can do this for parallelepiped cavities, closed tubes, and also spherical cavities.