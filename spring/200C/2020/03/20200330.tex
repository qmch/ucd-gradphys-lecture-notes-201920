We've completed our study of electrostatics, and now it's time to begin electrodynamics. We'll write the versions in materials to recap:
\begin{subequations}
    \begin{gather}
        \div \vec D = \rho_f\\
        \div \vec B = 0\\
        \curl \vec E = -\P{\vec B}{t}\\
        \curl \vec H = \vec J_f + \P{\vec D}{t}.
    \end{gather}
\end{subequations}

In statics, the total charge is
\begin{equation}
    \rho = \rho_f - \div \vec P,
\end{equation}
where $-\div \vec P$ is the polarization charge, and the total current is
\begin{equation}
    \vec J = \vec J_f + \curl \vec M,
\end{equation}
where $\curl \vec M$ is the magnetization current.%
    \footnote{Last quarter we also considered fictitious magnetic charges.}
    
In dynamics, there is an additional term in the total current. We need not only free current and magnetization current but also \emph{polarization current}:
\begin{equation}
    \vec J = \vec J_f + \curl \vec M + \P{\vec P}{t}.
\end{equation}
We'll also define the auxiliary fields
\begin{align}
    \vec D &= \epsilon_0 \vec E + \vec P = \epsilon \vec E,\\
    \vec H &= \frac{1}{\mu_0}\vec B - \vec M = \frac{1}{\mu} \vec B
\end{align}
Good dielectrics have $\epsilon/\epsilon_0 \sim 4$-$10$, while good magnets have permittivities of over $1000$.

Maxwell introduced an important correction to Amp\`ere's law. In statics,
\begin{equation}
    \curl \vec H = \vec J.
\end{equation}
It follows that since the divergence of a curl vanishes,
\begin{equation}
    \div \vec J = 0
\end{equation}
in statics. This makes sense, since it says that charge is not accumulating or depleting from anywhere. More generally, the continuity equation tells us that
\begin{equation}
    \div \vec J +\P{\rho_f}{t} =0,
\end{equation}
so since $\rho_f = \div \vec D$, we can write
\begin{equation}
    \div \paren{\vec J_f + \P{\vec D}{t}}=0
\end{equation}
and now we have a proper divergence-free quantity to put on the RHS of Ampere's law:
\begin{equation}
    \curl \vec H = \vec J_f + \P{\vec D}{t}
\end{equation}

\begin{exm}
    We have an infinite straight wire with cross-sectional area $\pi a^2$, and the wire has a break. We wish to find the displacement current in the gap. A current $I(t)=I_0 \cos \omega t$ flows in the wire.
    
    Since the charge can't jump the gap, we get a charge density $\sigma$ building up on each of the faces of the wire. Namely,
    \begin{equation}
        \vec E_\text{gap} = \frac{\sigma}{\epsilon_0}
    \end{equation}
    with
    \begin{equation}
        \sigma = \int \vec J dt = \int \frac{I(t)}{\pi a^2} dt,
    \end{equation}
    and then the displacement current is
    \begin{equation}
        \vec J_D = \P{\vec D}{t} = \P{\epsilon_0 \vec E}{t} = \epsilon_0 \P{\vec E_\text{gap}}{t} = \P{\sigma}{t} =\frac{I(t)}{\pi a^2}.
    \end{equation}
    We conclude that
    \begin{equation}
        I_D = \pi a^2 \vec J_D = I(t),
    \end{equation}
    so in fact the displacement current is equal to the current in the wire. Note we had to have a time dependence, or else our capacitor would charge up. At 20,000 V/cm, air breaks down and we get a spark.
\end{exm}

Faraday's law is the other important time-dependent Maxwell equation,
\begin{equation}
    \curl vec E = -\P{\vec B}{t}.
\end{equation}
Changing magnetic fields can induce an electromotive force (EMF), such that
\begin{equation}
    \epsilon = \oint_C \vec E \cdot d \vec l,
\end{equation}
such that $\vec E $ is the electric field at a line element $d\vec l$ of the circuit. This minus sign is important enough to have its own name, Lenz's law. Faraday's observations said that
\begin{equation}
    \epsilon = -K \frac{d\Phi}{dt},
\end{equation}
where $k=1$ in SI units. For a moving circuit (say, a wire loop), we see that the flux can change either if $\vec B$ changes or if the motion changes the flux through the circuit,
\begin{equation}
    \frac{d}{dt}\int_S \vec B \cdot \uv n dA = \int_S \P{\vec B}{t} \cdot \uv n d A + \oint (\vec B \times \vec v) \cdot d \vec l,
\end{equation}
Now the electromotive force is
\begin{equation}
    \oint_C \vec E \cdot d\vec l = -k \int_S \P{\vec B}{t}\cdot \uv n dA,
\end{equation}
where
\begin{equation}
    \vec E' = \vec E + k(v\times \vec B),
\end{equation}
with $\vec E'$ in the (co-)moving frame moving at velocity $\vec v$, and $\vec E$ measured in the lab frame. A conduction electron is basically at rest in the frame of the circuit, so it experiences a purely electric force, $\vec F = q\vec E'$ and so in the lab frame, $\vec F = q(\vec E + \vec v \times \vec B)$.

If the circuit is fixed in the lab frame, then
\begin{equation}
    \oint_C \vec E \cdot d\vec l = \int_S (\curl \vec E) \cdot \uv n dA = -\int_S \P{\vec B}{t} \cdot \uv n dA,
\end{equation}
so
\begin{equation}
    \curl \vec E = -\P{\vec B}{t}.
\end{equation}
