Last time, we started thinking about constructing the Green's function for a cylinder. We found that if we expanded the Green's function in a Fourier basis as
\begin{equation}
    G(\vec r,\vec r') = \frac{1}{2\pi^2} \sum_{m=-\infty}^\infty \int_0^\infty dk\, e^{im (\phi-\phi')} \cos k(z-z') G_m(\rho, \rho'; k),
\end{equation}
and via Laplace's (Poisson's) equation, the coefficients $G_m$ satisfy
\begin{equation}
    \frac{1}{\rho} \frac{d}{d\rho} \paren{\rho \frac{dG_m}{d\rho}}-\paren{k^2 + \frac{m^2}{\rho^2}} G_m = -\frac{1}{\epsilon_0} \frac{\delta(\rho-\rho')}{\rho},
\end{equation}
which is Bessel's equation away from $\rho = \rho'$ but with negative $k$.

One can then use the solutions to Bessel's equation to construct the Green's function in the regions $\rho < \rho'$ and $\rho > \rho'$. That is,
\begin{equation}
    G_m (\rho,\rho'; k) = \begin{cases}
        a_m I_m(k\rho), & \rho < \rho'\\
        b_m K_m(k\rho), & \rho > \rho'.
    \end{cases}
\end{equation}
That is, we consider $\rho'$ fixed and we want the solution for a delta-function charged cylinder. In fact, we write%
    \footnote{Equivalently this comes from the fact that the Green's function for a self-adjoint operator is symmetric in its arguments, cf. Arfken 8.2.}
\begin{equation}
    G_m (\rho,\rho'; k) = \begin{cases}
        a_m I_m(k\rho) K_m(k\rho'), & \rho < \rho'\\
        a_m K_m(k\rho) I_m(k\rho'), & \rho > \rho'
    \end{cases}
\end{equation}
by the matching condition of continuity at $\rho=\rho'$.

There's another matching condition, the discontinuity in the derivative about $\rho=\rho'$.%
    \footnote{This is very much like the delta-function potential in the Schr\"odinger equation.}
That is,
\begin{equation}
    \int_{\rho'-\Delta}^{\rho'+\Delta}d\rho \bkt{\frac{1}{\rho} \P{}{\rho}\paren{ \rho \P{G_m}{\rho}} - (k^2 + \frac{m^2}{\rho^2})G_m} = \int -\frac{1}{\epsilon_0} \frac{\delta(\rho-\rho')}{\rho} d\rho.
\end{equation}
The term proportional to $G_m$ is regular over the interval so it goes away; the other gives a derivative
\begin{equation}
    \P{G_m}{\rho}|{\rho \to \rho'{}^+} - \P{G_m}{\rho}|_{\rho \to \rho'{}^-} = -\frac{1}{\epsilon_0 \rho}.
\end{equation}
Now we explicitly compute the derivative.
\begin{equation}
    \P{G_m}{\rho}|_{\rho \to \rho'{}^-} = a_m k K_m(k\rho') I_m'(k\rho'), \quad \P{G_m}{\rho}|_{\rho \to \rho'{}^+} = a_m k K_m'(k\rho') I_m(k\rho'),
\end{equation}
so we find that
\begin{equation}
    a_m k \bkt{K_m'(\rho') I_m(k\rho') - I_m' (k\rho') K_m(k\rho') }= -\frac{1}{\epsilon_0 \rho'}.
\end{equation}
In fact, the LHS is just the Wronskian of the solutions to the Bessel equation. It is equal to 
\begin{equation}
    a_m k \bkt{K_m'(\rho') I_m(k\rho') - I_m' (k\rho') K_m(k\rho') } = a_m k \paren{-\frac{1}{k\rho'}},
\end{equation}
so we find that in fact
\begin{equation}
    a_m = \frac{1}{\epsilon_0}.
\end{equation}
Zangwill and Jackson both try to condense the notation by using $\rho_<$ and $\rho_>$ using the symmetry as
\begin{equation}
    G_m (\rho,\rho'; k) = \begin{cases}
        \frac{1}{\epsilon_0} I_m(k\rho) K_m(k\rho'), & \rho < \rho'\\
        \frac{1}{\epsilon_0} K_m(k\rho) I_m(k\rho'), & \rho > \rho'
    \end{cases} = \frac{1}{\epsilon_0} I_m(k\rho_<) K_m(k\rho_>).
\end{equation}

We've solved the Neumann boundary condition and found an expression which solves Laplace's equation away from the cylinder and has the right discontinuity on the wall ($\rho=\rho'$). What about the Dirichlet condition? We can add on a piece (another Bessel solution in $\rho$) to make this vanish at the wall $\rho =R$, i.e.
\begin{equation}
    G_m (\rho,\rho'; k) = \begin{cases}
        \displaystyle\frac{1}{\epsilon_0} I_m(k\rho) \bkt{K_m(k\rho')-I_m(k\rho') \frac{K_m(kR)}{I_m(kR)}}, & \rho < \rho'\\
        \displaystyle\frac{1}{\epsilon_0} K_m(k\rho) \bkt{I_m(k\rho')-I_m (k\rho)\frac{K_m(kR)}{I_m(kR)}}, & \rho > \rho'.
    \end{cases}
\end{equation}

%Last time, we said there were three ways of finding Green's functions.
\subsection*{Eigenfunction expansion}
We can do an eigenfunction expansion of the following (Sturm-Liouville) problem:
\begin{equation}
    -\nabla^2 \psi_n = \lambda_n \psi_n.
\end{equation}
This is a Schr\"odinger equation, as we know. And we will set $\psi=0$ on a boundary, specifically a cylinder. That is, we have a particle in a box. Suppose that we have some set of eigenfunctions $\set{\psi_n}$, suitably normalized (e.g. by the usual $L^2$ norm), and moreover the set is complete,
\begin{equation}
    \sum_n \psi_n(\vec r) \psi_n^*(\vec r') = \delta(\vec r- \vec r').
\end{equation}
We see that we've got a delta function; can we turn the LHS into something that looks like a Laplacian? We write down
\begin{equation}
    G_D(\vec r, \vec r') = \frac{1}{\epsilon_0} \sum_n \frac{\psi_n(\vec r) \psi_n^*(\vec r')}{\lambda_n}.
\end{equation}
We check that this is right by taking the Laplacian explicitly:
\begin{equation}
    \nabla^2 G_D = \frac{1}{\epsilon_0} \sum_n (-\psi_n(\vec r)) \psi^*_n(\vec r') = -\frac{1}{\epsilon_0} \delta(\vec r-\vec r').
\end{equation}
So this works! But the hard part is loaded into actually finding the eigenfunctions $\psi_n$.

Let's see what happens in cylindrical coordinates. We make the separable ansatz
\begin{equation}
    \psi(\rho,\phi,z) = R(\rho) \Phi(\phi) Z(z),
\end{equation}
and we write
\begin{equation}
    \nabla^2 \psi + \lambda \psi = 0.
\end{equation}
%I'm amazed that you have such strong feelings about variables. I do not understand.

Separation of variables yields
\begin{equation}
    \frac{1}{R}\frac{1}{\rho} \frac{d}{d\rho} \paren{\rho \frac{dR}{d\rho}} + \frac{1}{\rho^2} \frac{1}{\Phi} \frac{d^2 \Phi}{d\phi^2} = \underbrace{-\lambda -\frac{1}{Z} \frac{d^2 Z}{dz^2}}_{\equiv -k^2}.
\end{equation}
We pick our first separation constant to be $-k^2$ and find that
\begin{equation}
    (\lambda - k^2)Z = -\frac{d^2Z}{dz^2} \implies Z(z) = e^{\pm i \sqrt{\lambda-k^2} z}.
\end{equation}
Now we have
\begin{equation}
    \frac{1}{R}\frac{1}{\rho} \frac{d}{d\rho} \paren{\rho \frac{dR}{d\rho}} + \frac{1}{\rho^2} \frac{1}{\Phi} \frac{d^2 \Phi}{d\phi^2} = -k^2,
\end{equation}
and then we can multiply through by $\rho^2$ and set
\begin{equation}
    \frac{1}{\Phi} \frac{d^2\Phi}{d\phi^2} = -m^2 \implies \Phi(\phi) e^{\pm im\phi}.
\end{equation}
Finally, the last equation from separation of variables turns out to be Bessel's equation, so they have solutions
\begin{equation}
    R(\rho) = J_m(k\rho), N_m(k\rho).
\end{equation}
This time we want the oscillatory solutions in $\rho$ since we are in a confined region that does not extend to infinity. In fact we should throw away the $N_m$, since these diverge as $\rho\to 0$. The allowed values of $k$ come from the boundary condition on the side of the cylinder. Moreover we can choose
\begin{equation}
    Z(z) = \sin (\sqrt{\lambda-k^2}z),
\end{equation}
which has the property that it will vanish on the circular faces of the cylinder (say, of height $L$) provided that $\sqrt{\lambda-k^2} = \frac{n\pi}{L}$. We find that
\begin{equation}
    \psi_{mkn} = A_{mkn} \sin \frac{n\pi z}{L} e^{im\phi} J_m(k\rho),
\end{equation}
and to summarize, the condition of $m$ being discrete comes from periodicity in $\phi$; $k$ being discrete comes from setting the $J_m$s to be zero on the side of the cylinder; and $n$ comes from setting $Z(z)$ to be zero at $z=0$ and $z=L$. The eigenvalue $\lambda$ is then a function of $k$ and $n$, as
\begin{equation}
    \lambda = \frac{n^2\pi^2}{L} + k^2.
\end{equation}

We'll mostly skip Chapter 9 of Zangwill, which deals with steady-state currents without discussing magnetic fields. Next time we'll consider the Aharanov-Bohm effect, which is a fun and cool thing not in the book.

Let's just start thinking about steady-state charge and current distributions. In this limit, Maxwell's equations read
\begin{gather}
    \div \vec E = \frac{\rho}{\epsilon_0}, \quad \curl \vec E = 0,\\
    \div \vec B = 0, \quad \curl \vec B = \mu_0 \vec j.
\end{gather}
Since there is no time dependence, we can still use our tricks from potential theory and so on. The two equations dealing with $\vec E$ are totally unchanged.

We'll consider a version of Ohm's law which is as follows:
\begin{equation}
    \vec j = \sigma \vec E = \frac{1}{\rho} \vec E,
\end{equation}
where $\sigma$ is conductivity and $\rho$ is resistivity.