Last time, we discussed the generator of translations, which was the momentum operator. That is, the operator
\begin{equation}
    \hat T_{\vec a} = e^{-\frac{i\vec a \cdot \vec p}{\hbar}}.
\end{equation}
In the coordinate basis, the operator has the form
\begin{equation}
    e^{-\frac{ia}{hbar} \paren{-i\hbar \P{}{x}}} = e^{-a\P{}{x}},
\end{equation}
such that
\begin{equation}
    e^{-a \P{}{x} } \psi(x) = \psi(x-a).
\end{equation}
The translation group is abelian, i.e. its elements all commute with one another. All irreducible representations of the translation group are one-dimensional.

Our orthogonality relations moreover tell us that
\begin{equation}
    \int d^3r e^{-i\frac{\vec v \cdot \vec p'}{\hbar}} e^{+i\frac{\vec v \cdot \vec p}{\hbar}} =V \delta^{(3)} (\vec p- \vec p'),
\end{equation}
where $V$ indicates that this inner product is in a sense the volume of the space we're integrating over.

\subsection*{Rotations in 3D}
Rotations in 3D are a linear map $\RR^3\to \RR^3$, such that $R(\vec 0) = \vec 0$, while $R(\vec a + \vec b) = R \vec a + R \vec b.$ That is, we can add vectors and rotate their sum, or we can rotate each one first and then sum them.

If we write a coordinate system $\uv x, \uv y, \uv z: \hat e_1, \hat e_2, \hat e_3$ and some new vectors $\hat  e_1, \hat e_2, \hat e_3$.%
    \footnote{I've redone the exposition here.}

Let us posit that we can describe rotations by a matrix $R_{ij}$, so that $b'_i = R_{ij} b_j$. We would like the inner product to be preserved under rotations, such that
\begin{equation}
    a_i b_i = a_i' b_i'.
\end{equation}
Let us write this out explicitly.
\begin{align}
    a_i b_i = a_i' b_i' = R_{ij} a_j R_{ik} b_k.
\end{align}
But
\begin{equation}
    a_i b_i = a_j b_j = a_j \delta_{jk} b_k,
\end{equation}
so in fact
\begin{equation}
    R_{ij} R_{ik} = \delta_{jk} \implies R^T R = \II.
\end{equation}
These are \term{orthogonal matrices}. Notice that if $R^T R = \II$, then $\det (R^T R) = 1 \implies \det R = \pm 1$.

If we moreover require that $\det(R) = +1$ then we have the group of proper rotations, whereas $\det R = -1$ form the set of improper rotations. The proper rotations define a group $SO(3)$, which is the subgroup of $O(3)$ connected to the identity.

A rotation can be described by Euler angles, which we leave as an exercise. We can also do it by specifying an axis to rotate about $\uv n$ and an angle $\theta$. If we decompose a vector $\vec u$ into its components parallel $\vec u_\parallel$ and perpendicular $\vec u_\perp$ to $\uv n$, then
\begin{align}
    \vec u_\parallel &= \uv n \vec u \cdot \uv n,\\
    \vec u_\perp &= \vec u - \vec u_\parallel = (\uv n \times \vec u) \times \uv n.
\end{align}
It follows that
\begin{align}
    \vec u' &= \vec u_\parallel + \cos\theta (\uv n \times \vec u ) \times \uv n + \sin\theta(\hat u \times \vec u)\\
        &= \uv n (\vec u \cdot \uv n) + \cos\theta \bkt{\vec u - \uv n(\vec u \cdot \uv n)} + \sin\theta(\uv n \times \vec u)\\
        &= \cos\theta \vec u + \uv n (\vec u \cdot \uv n)(1-\cos\theta) + \sin\theta(\uv n \times \vec u).
\end{align}
One can check that the rotation matrix about the $\uv z$ axis has a simple form,
\begin{equation}
    R(\uv z,\theta) = \begin{pmatrix}
        R(\uv z,\theta) = \begin{pmatrix}
            \cos\theta & - \sin\theta & 0\\
            \sin\theta & \cos\theta & 0\\
            0 & 0 & 1
        \end{pmatrix}.
    \end{pmatrix}
\end{equation}
For infinitesimal $\theta$ we keep terms only to order $\theta$, so
\begin{equation}
    \vec u' = \vec u + \theta \uv n \times \vec u.
\end{equation}
From the matrix point of view, let us write
\begin{equation}
    R_\epsilon = \II + \epsilon A.
\end{equation}
If $R$ is to be orthogonal, then
\begin{equation}
    \II = R_\epsilon^T R = (1+\epsilon A^T) (1+ \epsilon A) = \II + \epsilon (A^T + A) + O(\epsilon^2) \implies A^T + A =0.
\end{equation}
Hence the generator $A$ is antisymmetric. The most general $3\times 3$ antisymmetric matrix takes the form
\begin{equation}
    A= \begin{pmatrix}
        0 & -a_3 & a_2\\
        a_3 & 0 & -a_1\\
        -a_2 & a_1 & 0
    \end{pmatrix} = a_i \mathscr{J},
\end{equation}
with
\begin{equation}
    \mathscr{J}_1 =\begin{pmatrix}
        0 & 0 & 0\\
        0 & 0 & -1\\
        0 & 1 & 0
    \end{pmatrix}, \mathscr{J}_2 = \begin{pmatrix}
        0 & 0 & 1\\
        0 & 0 & 0\\
        -1 & 0 & 0
    \end{pmatrix}, \mathscr{J}_3 = \begin{pmatrix}
        0 & -1 & 0\\
        1 & 0 & 0\\
        0 & 0 & 0
    \end{pmatrix}
\end{equation}
%
For infinitesimal $\theta=\epsilon$, we have
\begin{equation}
    R(\uv n, \epsilon) = I+\epsilon \uv n \cdot \vec{\mathscr{J}}.
\end{equation}

Under matrix multiplication, these generators obey an anticommutator (a Lie bracket, actually):
\begin{equation}
    [\mathscr{J}_i, \mathscr{J}] = \epsilon_{ijk} \mathscr{J}_k.
\end{equation}
Note that the matrices actually have the elements $(\mathscr{J}_k)_{ij} = \epsilon_{kij}$, so
\begin{align}
    \bkt{(\vec a \cdot \mathscr{J})\vec u}_i &= (\vec a \cdot \mathscr{J})_{ij} u_j\\
        &= a_k (\mathscr{J}_k)_{ij} u_j\\
        &= \epsilon_{kij} a_k u_j = \vec a \times \vec u.
\end{align}
The anticommutator tells us that this group is non-abelian--- it has nontrivial commutators.

We can find a general rotation by noticing that
\begin{equation}
    \frac{dR(\uv n,\theta)}{d\theta} = \lim_{\epsilon\to 0} \frac{R(\uv n, \theta + \epsilon) - R(\uv n, \theta)}{\epsilon} = \lim_{\epsilon\to 0} \paren{\frac{R(\uv n, \epsilon) - 1}{\epsilon}} R(\uv n, \theta) = (\uv n \cdot \mathscr{J}) R(\uv n, \theta).
\end{equation}
We conclude that
\begin{equation}
    R(\uv n, \theta) = e^{\theta \uv n \cdot \mathscr{J}}.
\end{equation}
%
As it turns out,
\begin{equation}
    R(\vec a \times \vec u) =\det R (R\vec a) \times (R \vec u).
\end{equation}

%Aaaaand I'm out. Have fun, y'all.