Today we'll discuss using contour integrals to compute infinite sums. We'd like to sum a series like
\begin{equation}
    \sum_{n} a_n,
\end{equation}
and we want to write it in terms of a contour integral. Consider the function $\cot z = \frac{\cos z}{\sin z}$. The function $\sin z$ has the product expansion
\begin{equation}
    \sin z = z \prod_{n=1}^\infty \paren{1-\frac{z}{n\pi}} \paren{1+\frac{z}{n\pi}},
\end{equation}
so we have (nondegenerate) zeroes at integer $\pi$ and therefore $\cot z$ has simple poles at integer $\pi$.%
    \footnote{The degeneracy of a zero is easily counted by taking derivatives and evaluating the function at that zero until the net result is nonzero. For instance, $f(x)=(x-1)^2$ has a double zero at $x=1$. But $f(1)=0$ and $f'(1)=2(x-1)|_{x=1}=0$; only $f''(1)=2\neq 0$.}
Thus the residue at such poles is
\begin{equation}
    \lim_{z\to n\pi}(z-n\pi) \cot z = \frac{(z-n\pi)\cos z}{\sin z}.
\end{equation}
We can set a normalization so that $\cot \pi z$ has poles at integer $n$. Now consider the function
\begin{equation}
    f(z) \cot(\pi z),
\end{equation}
where $f(z)$ does not have any poles at real integer values. Suppose that $f(z)$ has some poles in the complex plane, however, at $\set{z_i}$. Then if we take the contour integral over a circle of radius $R$ centered on the origin and take $R\to \infty$, then
\begin{equation}
    \oint f(z) \cot (\pi z) dz = 2\pi i \sum_{n=-\infty}^\infty f(n) + 2\pi i \sum_{z_i} \text{Res} f(z)\cot \pi z.
\end{equation}
If $f$ decays sufficiently fast as $z\to \infty$, then the contour integral over the circle at infinity is zero, and we conclude that
\begin{equation}
    \sum_{n=-\infty}^\infty f(n) = -\sum_{z_i} \text{Res} f(z)\cot \pi z.
\end{equation}
\begin{exm}
    Suppose we wish to sum the series
    \begin{equation}
        S=\sum_{n=1}^\infty \frac{1}{n^2+a^2}.
    \end{equation}
    Notice that the summand is even in $n$, so
    \begin{equation}
        S= \sum_{n=-1}^\infty \frac{1}{n^2+a^2}.
    \end{equation}
    We can easily add on the $n=0$ term, which is just $1/a^2$. Thus
    \begin{equation}
        2S+\frac{1}{a} = \sum_{n=-\infty}^\infty \frac{1}{n^2+a^2},
    \end{equation}
    which is related to the contour integral
    \begin{equation}
        \oint_{C_R} dz\frac{1}{z^2+a^2} \cot(\pi z).
    \end{equation}
    This has poles everywhere we want, but it also has poles at $z=\pm ia$. So the value of this integral is
    \begin{equation}
        2\pi i \paren{\sum_{n=-\infty}^\infty \frac{1}{n^2+a^2} + 2\frac{\cot (\pi i a)}{2ia}}=0,
    \end{equation}
    and so
    \begin{equation}
        \sum_{-\infty}^\infty \frac{1}{n^2+a^2} = \frac{\pi}{a} \coth(\pi a).
    \end{equation}
\end{exm}
\begin{exm}
    Consider the sum
    \begin{equation}
        \sum_{n=1}^\infty \frac{1}{n(n+1)} = \sum_{n=1}^\infty \paren{\frac{1}{n}-\frac{1}{n+1}}.
    \end{equation}
    There's a somewhat pedestrian way to sum this series. If we take the partial sums $\sum_{n=1}^N$ then these sums are
    \begin{equation}
        1+ \frac{1}{2} + \frac{1}{3}+\dots + \frac{1}{N} - \paren{\frac{1}{2} + \frac{1}{3}+\dots + \frac{1}{N+1}}=1-\frac{1}{N+1},
    \end{equation}
    and as $N\to \infty$ we see the sum is just $1$.
    
    Let's do this a tricky way. By shifting $n\to -n$ and changing the index, we can see that
    \begin{equation}
        \sum_{n=-2}^\infty \frac{1}{n(n+1)}=\sum_{n=1}^\infty \frac{1}{n(n+1)},
    \end{equation}
    our original sum. We can almost extend this to the full infinite sum $\sum_{n=-\infty}^\infty$, but the $n=0$ and $n=-1$ terms are no good. We can write
    \begin{equation}
        2S= \sum_{-\infty}^\infty{}' \frac{1}{n(n+1)},
    \end{equation}
    where the prime indicates we must omit $n=0$ and $n=-1$. These are poles, which suggests that we have picked up double-poles at $n=0,n=-1$. That is,
    \begin{equation}
        0=\oint dz \frac{1}{z(z+1)} \cot \pi z.
    \end{equation}
    The residue at $z=0$ is the usual limit
    \begin{equation}
        \lim_{z\to 0} z^2 \cot \pi z
    \end{equation}
    and $z=-1$ is similar. Hence
    \begin{equation}
        \sum_{-\infty}^\infty{}' \frac{1}{n(n+1)} = -\text{Res}\paren{\frac{1}{z(z+1)}\cot(\pi z}_{z=0}-\text{Res}\paren{\frac{1}{z(z+1)}\cot(\pi z}_{z=-1}.
    \end{equation}
    This is precisely analogous to having dipoles rather than monopoles at some integer locations on the real line.
\end{exm}
%is 1/5 minus one sixth zero? Yes? No, it's 1/30!

\subsection*{Schwartz reflection principle}
The Schwartz reflection principle tells us how to take complex conjugates of various functions. We have seen previously that $f^*(z)= f(z^*)$, but in fact there are conditions. Consider
\begin{equation}
    \bkt{(z-x_0)^n}^* = (z^*-x_0)^n,
\end{equation}
where $x_0\in \RR$. When we take linear combinations $\sum a_n(z-z_0)^n$, we might have to complex conjugate the $a_n$s as well. The Schwartz reflection principle provides sufficient conditions so this does not happen.

Consider $f(z)$ analytic. Hence we can Taylor expand $f$ about a point $x_0$ on the real line,
\begin{equation}
    f(z) = \sum \frac{f^{(N)}(x_0)}{n!}(z-x_0)^n,
\end{equation}
where $z_0$ is real. Moreover, suppose that for real arguments $x_0\in \RR$, $f^*(x_0)=f(x_0)$, i.e. $f$ is real-valued for real arguments. Hence all the $f^{(n)}(x_0)$ are also real. This must be the case since the $(z-x_0)^n$ factors are all linearly independent. Hence
\begin{equation}
    f^*(z) = \sum \frac{f^{(n)}(x_0)}{n!}(z^*-x_0)^n = f(z^*).
\end{equation}

