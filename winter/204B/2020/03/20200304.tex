\begin{quote}
    \textit{``I like all things if they're valuable, okay? I collect them.''}
    
    ---Nemanja Kaloper
\end{quote}

Building off last time, we can define 3-dimensional equivalents of the Fourier transform,
\begin{equation}
    f(\vec x) = \int \frac{d^d\vec k}{(2\pi)^{d/2}}g(\vec k) e^{-i\vec k \vec x}
\end{equation}
and
\begin{equation}
    g(\vec k) = \int \frac{d^d\vec x}{(2\pi)^{d/2}}f(\vec x) e^{i\vec k \cdot \vec x}.
\end{equation}

\begin{exm}
    Consider the Yukawa potential,
    \begin{equation}
        f(r) = \frac{e^{-\alpha r}}{r}=\frac{e^{-r/m}}{r},
    \end{equation}
    where $\alpha=m >0$. We may do this in order to ensure convergence. Consider now its Fourier transform,
    \begin{equation}
        \int d^3x \frac{e^{-\alpha|\vec x|}}{|\vec x|} e^{-i|\vec k| |\vec x| \cos \theta}.
    \end{equation}
    Since we integrate over all space and all angles, WLOG we can take $\vec k$ to point along the $\uv z$ direction in order to perform the integral. Notice that this integral retains an azimuthal symmetry, so it's most natural to transform to spherical coordinates. This integral becomes
    \begin{equation}
        2\pi \int_0^\infty dr r^2 \int_0^\pi d\theta \sin\theta \frac{e^{-\alpha|\vec x|}}{|\vec x|} e^{-i|\vec k| |\vec x| \cos \theta}.
    \end{equation}
    The book resorts to expanding in spherical harmonics, but there's really no need to do it. We have $d\sin\theta = d(-\cos\theta)$ so the angular integral is easy to do. Our integral simplifies to
    \begin{align*}
        g(\vec k) &= 2\pi \int_0^\infty dr \, re^{-\alpha r} \int_{-1}^1 d(\cos\theta) e^{-ikr(\cos\theta)}\\
            &= \frac{2\pi i}{k} \int_0^\infty dr\, e^{-\alpha r} \paren{e^{-ikr} - e^{+ikr}}\\
            &= \frac{2\pi i}{k} \int_0^\infty dr \paren{e^{-(\alpha +ik)r} - e^{-(\alpha-ik)r}}.
    \end{align*}
    This last expression is similar to what we computed last time. The $e^{\pm ikr}$ phase doesn't affect convergence, since $e^{-\alpha r}|e^{ikr}|$ converges absolutely. Thus
    \begin{equation}
        g(k) = \frac{2\pi i}{k} \bkt{\frac{1}{\alpha+ik} - \frac{1}{\alpha-ik}} = \frac{2\pi i}{k} \frac{-2ik}{\alpha^2+k^2} = \frac{4\pi}{\alpha^2+k^2}.
    \end{equation}
    The Fourier transform of the Yukawa potential has turned out to be a Lorentzian in $k$-space.%
        \footnote{See \url{http://mathworld.wolfram.com/LorentzianFunction.html}.}
    However, note that if the mass is zero, we lose the factor $e^{-\alpha|\vec x|}$ which gave us convergence. What happens now? Our integral becomes
    \begin{equation}
        \frac{2\pi i}{k} \int_0^\infty dr \paren{e^{-ikr} - e^{+ikr}} = -\frac{2\pi i}{k} \paren{\int_0^L + \int_0^{-L}}dr e^{ikr}.
    \end{equation}
    The lower limit gives us the correct answer, but we have to understand the behavior at infinity. Inserting the regularizing factor $e^{-\alpha|x|}$ allows us to actually cut off the behavior at infinity (at large distances $x$) and then take the $\alpha\to 0$ limit.
    
    Hence we we can say that (restoring factors of $2\pi$)
    \begin{equation}
        g(k) = \frac{1}{(2\pi)^{3/2}} \frac{4\pi}{\alpha^2 +k^2}
    \end{equation}
    and in the $\alpha\to 0$ limit,
    \begin{equation}
        g(k) \to \frac{1}{(2\pi)^{3/2}} \frac{4\pi}{k^2}.
    \end{equation}
    Let us take the inverse transform and confirm that it all works out. That is, we shall integrate
    \begin{align*}
        \int d^3x \frac{e^{-i|\vec k||\vec x|\cos\theta}}{|\vec x|^2} &= 2\pi \int_0^\infty dr \int_{-1}^1 d(\cos\theta) e^{-ikr(\cos\theta)}\\
            &= \frac{2\pi i}{k} \int_0^\infty dr \frac{1}{r} \paren{e^{-ikr} - e^{ikr}}\\
            &= -\frac{2\pi i}{k} \int_0^\infty \frac{dr}{r} \paren{e^{+ikr} - e^{-ikr}}.
    \end{align*}
    Now we can flip the sign in the exponent of the second term and get
    \begin{equation}
        -\frac{2\pi i}{k} \bkt{\int_0^\infty \frac{dr}{r} e^{ikr} - \int_0^{-\infty} \frac{dr}{r} e^{+ikr}},
    \end{equation}
    which we recognize as a single principal value integral,
    \begin{equation}
        -\frac{2\pi i}{k} \Int \frac{dr}{r} e^{ikr}.
    \end{equation}
    Moreover the combination $dr/r$ is scale-invariant, as are the limits of integration, so we can reparametrize to $\Int \frac{dz}{z} e^{-iz}.$ Performing the integral gives us our final answer,
    \begin{equation}
        \frac{4\pi^2}{k},
    \end{equation}
    which upon restoring $2\pi$s gives us our original Coulombic potential as promised.
\end{exm}

Let's mention some properties of the Fourier transform. If we have a function $f(\vec x)$, let us denote its Fourier transform as
\begin{equation}
    \bkt{f(\vec x)}^T = \frac{1}{(2\pi)^{3/2}}\int d^3 \vec x e^{i\vec k \cdot \vec x}.
\end{equation}
Then the Fourier transform of the shifted function is
\begin{equation}
    \bkt{f(\vec r- \vec R}^T = e^{i\vec k \cdot \vec R} g(\vec k).
\end{equation}
Similarly the Fourier transform of the rescaled argument is
\begin{equation}
    \bkt{f(\alpha \vec r)}^T = \frac{1}{\alpha^3} g\paren{\frac{\vec k}{\alpha}},
\end{equation}
where the overall scaling factor comes from the integration measure $d^3x$. Inversion gives
\begin{equation}
    \bkt{f(-\vec r)}^T = g(-\vec k),
\end{equation}
setting $\alpha =-1$. Note there's no overall minus sign because both the measure and the limits of integration flip. Complex conjugation and inversion gives
\begin{equation}
    \bkt{f^*(-\vec r)}^T = g^*(\vec k),
\end{equation}
while taking a derivative brings down an $ik$:
\begin{equation}
    \bkt{\grad f(\vec r)}^T = -i\vec k g(\vec k),
\end{equation}
and doing the Laplacian gives
\begin{equation}
    \bkt{\nabla^2 f(\vec x)}^T = -k^2 g(\vec k).
\end{equation}

%I like all things if they're valuable, okay? I collect them.
%Imagine two guys sitting in two boats on a lake throwing snowballs at each other.