\begin{quote}
    \textit{``If you're going to try to violate decoupling, don't try to do it with something as stupid as continuous functions.''
    ``Bastards.''
    ``That's correct.''
    }
    
    ---Nemanja Kaloper and Mark Samuel Abbott
\end{quote}

Last time, we found the explicit formula for the partial sums of a Fourier series to be
\begin{equation}
     f_N(x) = \frac{1}{2\pi} \int_{-\pi}^\pi dt f(t) \bkt{\frac{\sin((N+1/2)(x-t))}{\sin \frac{x-t}{2}}}.
 \end{equation}
 Let us consider a Heaviside step function,
 \begin{equation}
     f(x) = \frac{h}{2}(2\Theta(x) - 1).
 \end{equation}
 We found the Fourier series to be
 \begin{equation}
     f(x) = \frac{2h}{\pi} \sum_{n=0}^\infty \frac{\sin \bkt{(2n+1)x}{2n+1}},
 \end{equation}
 and we'd like to study what happens when the sum is finite rather than infinite. That is, let us plug in $f(t)$. Then
 \begin{align}
     f_N(x) &= \frac{1}{2\pi} \int_{-\pi}^\pi dt \frac{h}{2}(2\Theta(t) - 1) \bkt{\frac{\sin((N+1/2)(x-t))}{\sin \frac{x-t}{2}}}\\
        &= \frac{h}{4\pi} \bkt{-\int_{-\pi}^0 dt \frac{\sin((N+1/2)(x-t))}{\sin \bkt{\frac{x-t}{2}}} + \int_0^\pi dt \frac{\sin((N+1/2)(x-t))}{\sin\bkt{ \frac{x-t}{2}}}}.
 \end{align}
 Since this is a finite series and the function is piecewise continuous, we can now manipulate the partial sums and exchange order of integration and summation as we like, then take the limit of the interpolating functions $\lim_{N\to \infty}f_N(x)$ to get the infinite sum.
 
 
 Note that
 \begin{equation}
     \frac{1}{2\pi} \int_{-\pi}^\pi dt \frac{\sin \bkt{(N+1/2)(x-t)}}{\sin\bkt{\frac{x-t}{2}}} = 1.
 \end{equation}
 This is just the integral of (the limiting series of) a delta function. Why? This was just the sum
 \begin{equation}
     \sum_{n=-N}^N e^{in(x-t)},
 \end{equation}
 and if we integrate from $-\pi$ to $\pi$, then we have
 \begin{equation}
     e^{inx} \int_{-\pi}^\pi dt e^{-int} = 2\pi \delta_{n,0},
 \end{equation}
 so the other terms in the sum drop out.
 %If you're going to try to violate decoupling, don't try to do it with something as stupid as continuous functions.
 %Bastards.
 %That's correct.
 %Did you like the diffraction grating problem on the prelim? With the path difference? That was mine. It is simple. But it is tricky.
 Note that our integral is not over a whole period. The behavior of the integral is under $t\to -t$, so for generic $x$, this function will be neither odd nor even.
 
 We can rewrite the terms in our integrals in terms of
 \begin{equation}
     x-t = -s \text{ or } x-t = s
 \end{equation}
 to write
 \begin{align}
     f_N(x) &= \frac{h}{\pi} \bkt{-\int_{-\pi+x}^{+x} ds \frac{\sin((N+1/2)(-s))}{\sin \bkt{\frac{-s}{2}}} + \int_{-x}^{\pi-x} ds \frac{\sin((N+1/2)(-s))}{\sin\bkt{ \frac{-s}{2}}}}\\
        &=\frac{h}{4\pi} \bkt{-\int_{-\pi+x}^{+x} ds \frac{\sin((N+1/2)s)}{\sin \bkt{\frac{s}{2}}} + \int_{-x}^{\pi-x} ds \frac{\sin((N+1/2)(s))}{\sin\bkt{ \frac{s}{2}}}}\\
        &=\frac{h}{4\pi} \bkt{+\int_{-\pi+x}^{+x} ds \frac{\sin((N+1/2)s)}{\sin \bkt{\frac{s}{2}}} - \int_{-\pi - x}^{-x} ds \frac{\sin((N+1/2)(s))}{\sin\bkt{ \frac{s}{2}}}}.
 \end{align}
 We can shuffle limits of integration a bit to get
 \begin{equation}
     \frac{h}{4\pi}\bkt{+\int_{-x}^{+x} ds \frac{\sin((N+1/2)s)}{\sin \bkt{\frac{s}{2}}} - \int_{-\pi - x}^{-\pi + x} ds \frac{\sin((N+1/2)(s))}{\sin\bkt{ \frac{s}{2}}}}
 \end{equation}
 by noting that the relative sign allowed us to rewrite the definite integral 
 \begin{equation}
     -\Phi(-x) + \Phi(-\pi - x) + \Phi(x) - \Phi(-\pi +x) = (-\Phi(-x)+\Phi(x)) +(\Phi(-\pi-x) - \Phi(-\pi +x)).
 \end{equation}
 Now what happens as $x\to 0$? Our measure $ds$ gets vanishingly small, and if the integrand were finite on that range, the integral would vanish. But the $\sin(s/2)$ does something more interesting. The second term is surely bounded, since $\sin((N+1/2)s)$ is bounded around $s=-\pi$, and $\sin(s/2) \to \sin(-\pi/2)$ is perfectly well-behaved. So indeed the second integral is zero. More precisely, it is proportional to the interval length $2x$, which goes to zero as $x\to 0$.
 
 The first one is different. $\sin(0)$ is zero, so the denominator will cause the integrand to diverge as $s\to 0$. The numerator $\sin((n+1/2)s)$ also goes to zero, so we get an indeterminate ratio which we can work out with L'h\^opital's rule.
 \begin{equation}
    \frac{h}{4\pi} \int_{-x}^x \frac{\sin((N+1/2)s)}{\sin(s/2)} ds = \frac{h}{4\pi} \int_{-x}^x ds \frac{(N+1/2)s}{(s/2)}.   
 \end{equation}
 
 The factors of $s$ cancel. Now we must be careful about the order of limits. If we hold $N$ fixed, then we get
 \begin{equation}
     \frac{h}{4\pi} 2(N+1/2) (2x),
 \end{equation}
 which vanishes as $x\to 0$. But then all the functions in the series are zero, which tells us that the value of the step function is the limit as $N\to \infty$ after $x\to 0$, i.e. 
 \begin{equation}
     \Theta(0)-1/2=0,
 \end{equation}
 the average of its value on either side.
 
 If we look at the Fourier series approximation of the step function, we find that the approximation overshoots the step. This is called the Gibbs phenomenon. Curiously, the overshoot doesn't improve with more terms; in fact, it approaches a limiting value of about $18\%$. The reason for this is simply that we're trying to approximate a discontinuous function with a smooth one. We overshoot because the derivative cannot change too quickly. Next time, we will study the behavior of
 \begin{equation}
    \frac{h}{2\pi} \int_0^x \frac{\sin\bkt{(N+1/2)s}{\sin(s/2)}} dx,
 \end{equation}
 and we can study the behavior of this integral in terms of a new variable
 \begin{equation}
     \zeta = (N+\frac{1}{2})s.
 \end{equation}
 Then we have an integral
 \begin{equation}
     \frac{h}{2\pi} \int_0^{(N+1/2)x} \frac{\sin(\zeta)}{\sin(\zeta/(2N+1))} d\zeta \approx \frac{h(2N+1)}{2\pi} 
     \frac{h}{2\pi} \int_0^{(N+1/2)x} \frac{\sin(\zeta)}{\zeta} d\zeta.
 \end{equation}