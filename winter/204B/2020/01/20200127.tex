\begin{quote}
    \textit{``Suppose you have a transparent beach ball. And you take one of these laser pointers that fools like to shine at airplanes and such things, and you shine it through your beach ball.''}
    
    ---Nemanja Kaloper
\end{quote}

Today's material is drawn from Arfken 6th edition, since the 7th edition has some sort of error we'll consider shortly. Consider a complex-valued function $f(z): \CC\to \CC$. Note that complex functions need not be one-to-one, and in fact they can even be multivalued if we're not careful (cf. our discussion of the function $\ln z, z\in \CC$).

We can define a \term{Banach space} as a set such that for any two points $z_1,z_2\in \CC$, there exists a smooth function $f:[0,1] \to \CC$ such that $f(0) = z_1, f(1) =z_2$.%
    \footnote{This seems to be a statement of the vector space being complete, in the sense we can define limits.}
This allows us to talk abut derivatives of functions:
\begin{equation}
    f'(z) = \lim_{\Delta z \to )} \frac{f(z+\Delta z) - f(z)}{\Delta z},
\end{equation}
and the derivative $\frac{df}{dz}$ through a point must be unique and independent of the path.

The existence and uniqueness of the derivative for complex functions is related to a set of constraints known as the Cauchy-Riemann conditions. That is,
\begin{equation}
    \exists f'(z) \iff \P{u}{x} = \P{v}{y}, \P{u}{y} = - \P{u}{x},
\end{equation}
where our complex function is
\begin{equation}
    f = u+ iv,\quad u, v: \CC \to \RR,
\end{equation}
and points in the domain are given by
\begin{equation}
    z= x+ iy, \quad x,y \in \RR.
\end{equation}
Let us consider two possible displacements,
\begin{equation}
    \Delta z = \Delta x, \quad \Delta z = i\Delta y.
\end{equation}
In the first case, the derivative takes the form
\begin{equation}
    \frac{df}{dz} = \frac{u(x+\Delta x, y) + i v(x+ \Delta x,y) - u(x,y) - i v(x,y)}{\Delta x} \to \P{u}{x} + i \P{v}{x}
\end{equation}
in the limit as $\Delta x \to 0$. In the second case, we instead have
\begin{equation}
    \frac{df}{dz} = \frac{u(x,y+\Delta y)+iv(x,y+\Delta y) - u(x,y) - i v(x,y)}{i\Delta y} = -i\P{u}{y} + \P{v}{y}.
\end{equation}
Now we equate real and imaginary parts and find that
\begin{equation}
    \P{u}{x} = \P{v}{y}, \quad \P{v}{x} = -\P{u}{y}.
\end{equation}
Hence the existence of the derivative implies Cauchy-Riemann.

We shall now prove the converse, that Cauchy-Riemann implies the existence of the derivative. Suppose we have a path through a point $z$, and we are given the appropriate derivatives of $u$ and $v$ for some function $f$. Then
\begin{equation}
    \Delta f = \paren{\P{u}{x}\Delta x + \P{u}{y} \Delta y} + i \paren{\P{v}{x} \Delta x + \P{v}{y} \Delta y}.
\end{equation}
Hence
\begin{equation}
    \frac{\Delta f}{\Delta z} = \frac{\paren{\P{u}{x}\Delta x + \P{u}{y} \Delta y} + i \paren{\P{v}{x} \Delta x + \P{v}{y} \Delta y}}{\Delta x + i \Delta y}.
\end{equation}
Let us now apply Cauchy-Riemann and eliminate the $v$ derivatives. That is,
\begin{equation}
    \frac{\Delta f}{\Delta z} = \frac{\paren{\P{u}{x}\Delta x + \P{u}{y} \Delta y} + i \paren{-\P{u}{y} \Delta x + \P{u}{x} \Delta y}}{\Delta x + i \Delta y} = \P{u}{x}-i \P{u}{y}.
\end{equation}
It follows that all the path dependence in $\Delta x$ and $\Delta y$ drops out, so the derivative exists and moreover it is path-independent.
We could also write
\begin{equation}
    \frac{df}{dz} = \P{u}{x} +i\P{v}{x},
\end{equation}
using Cauchy-Riemann.

Now consider the function
\begin{equation}
    z^2 = (x+iy)^2 = \underbrace{x^2 -y^2}_u + i\underbrace{2xy}_v.
\end{equation}
We see that
\begin{equation}
    \P{u}{x} = 2x = \P{v}{y}, \quad \P{u}{y} = -2y = -\P{v}{x}.
\end{equation}
So this function is really analytic, as we might expect. In contrast,
\begin{equation}
    |z| = \sqrt{x^2+y^2}
\end{equation}
is not analytic, since $v=0$ for this function but $\P{u}{x} = \frac{x}{\sqrt{x^2+y^2}} \neq 0$ in general.

Since we have
\begin{equation}
    f'(z) = \P{u}{x} + i \P{v}{X},
\end{equation}
we can now import all our results from regular calculus of real variables. The chain rule and the product rule all follow in the usual way.

Suppose moreover that the second derivatives of $u$ and $v$ exist. Consider deriving the first with respect to $x$, and the second with respect to $y$. That is,
\begin{equation}
    \frac{\p^2 u}{\p x^2} = \frac{\p^2 v}{\p x \p y}, \quad \frac{\p^2 u}{\p^2 y} = -\frac{\p^2 v}{\p x \p y}.
\end{equation}
By the equality of mixed partials, we learn that
\begin{equation}
    \frac{\p^2 u}{\p x^2} = - \frac{\p^2 u}{\p y^2},
\end{equation}
or equivalently
\begin{equation}
    \Delta u = 0, \Delta v = 0,
\end{equation}
since we can perform the same argument for $v$ as well. This states exactly that $u$ and $v$ are \emph{harmonic functions}; they satisfy the 2D Laplace equation.

Consider this in the form
\begin{equation}
    0 = \delta u = \P{u}{x} \delta x  + \P{u}{y} \delta y.
\end{equation}
This defines a trajectory along an equipotential. Then
\begin{equation}
    \left.\frac{\Delta y}{\Delta x}\right|_\text{u=\text{constant}} = -\frac{\p_x u}{\p_y u} = + \frac{\p_y v}{\p_x v}.
\end{equation}
This tells us that in fact $u$ and $v$ are related in a special way---in fact, the equipotentials of $u$ and $v$ are orthogonal.
%When people want to pretend to be not shy but cool, they just look at the blank screen of their phone.

To understand the idea of a ``point at infinity,'' let us place a two-sphere on the complex plane, with the south pole at the origin of the complex plane. Suppose we trace a line from the north pole to a point on the sphere, and then extend that line until it intersects the complex plane. Clearly, this is a one-to-one mapping between the plane and the sphere. But where is the point at the north pole mapped to? Effectively, it is mapped to all the points on a circle of infinite radius.%
    \footnote{This construction is better known as the Riemann sphere.}
%Suppose you have a transparent beach ball. And you take one of these laser pointers that fools like to shine at airplanes and such things, and you shine it through your beach ball.

We can then define a Riemann sum on some path $f(z)$ starting at $z_0$ and ending at $z$. We can discretize the path as $\set{z_i}$ and sample along the way as
\begin{equation}
    \sum_i f(z_i) \Delta z.
\end{equation}
If the limit $\Delta z \to 0$ of this sum exists, then we say that
\begin{equation}
    \lim_{\Delta z  \to 0} \sum_i f(z_i) \Delta z \equiv \int_{z_0}^z f(z') dz'.
\end{equation}
In fact, we can write
\begin{equation}
    \int_C f(z) dz = \int (u+iv) (dx+idy) = \int(udx - vdy) + i \int(u dy + vdx).
\end{equation}