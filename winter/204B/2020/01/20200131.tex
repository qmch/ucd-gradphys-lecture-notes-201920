\begin{quote}
    \textit{``We don't know what this does on the circle. Except we *do* know. We just haven't realized that we know.''}
    
    ---Nemanja Kaloper
\end{quote}

Last time, we discussed Cauchy's integral theorem. There's a way to deal with contour integrals near singular points. The contour integral over two contours that can be continuously deformed into one another without crossing singularities have the same value. This is precisely the Gauss's law statement that the flux through two surfaces that can be continuously deformed into each other is equal, provided that we have not enclosed any new charge.

Let's take a function
\begin{equation}
    \frac{f(z)}{z-z_0},
\end{equation}
where $z_0\in \CC$ is some point in the complex plane and $f(z)$ is analytic in some domain of analyticity including $z_0$. Consider the integral on a contour within the domain of analyticity which contains $z_0$,
\begin{equation}
    \oint_C \frac{f(z)}{z-z_0}.
\end{equation}
This integral is nonzero, since it includes a singular point.

How can we compute its value? Consider as our contour $C_\rho$, a circle centered on $z_0$ of radius $\rho$. Each point on the circle is $z= z_0+ \rho e^{i\phi}$. Now we can compute the contour integral explicitly. By flux conservation (this is an isolated singularity), any other contour enclosing $z_0$ has the same integral as over $C_\rho$. THe integral is
\begin{equation}
    \oint_{C_\rho} dz \frac{f(z)}{z-z_0} = i\int_0^{2\pi} d\phi \rho e^{i\phi} \frac{f(z_0 + \rho e^{i\phi})}{\rho e^{i\phi}} = i \int_0^{2\pi} d\phi f(z_0 + \rho e^{i\phi}).
\end{equation}
If we now take the limit as $\rho\to 0$, we can use the fact that $f$ is analytic and therefore continuous at $z_0$ so that in fact
\begin{equation}
    \oint_{C_\rho} dz \frac{f(z)}{z-z_0} = 2\pi i f(z_0).
\end{equation}
%This sounds by itself really strong and cool. Well, guess what. It gets much better now.
We can combine this with our previous knowledge of integrals of analytic functions to say that the integral $\oint_C dz \frac{f(z)}{z-z_0}$ is zero if $z_0$ is outside the contour, and $2\pi i f(z_0)$ if it is inside.

Moreover, let us write
\begin{equation}
    f(z_0+\delta z) = \frac{1}{2\pi i} \oint dz \frac{f(z)}{z-(z_0+\delta z)}.
\end{equation}
Since
\begin{equation}
    f(z_0) = \frac{1}{2\pi i} \oint dz \frac{f(z)}{z-z_0},
\end{equation}
it follows that expanding to first order and taking the limit $\delta z \to 0$, we get
\begin{equation}
    f'(z_0) = \lim_{\delta z \to 0} \frac{f(z_0+\delta z) - f(z_0)}{\delta z} = \frac{1}{2\pi i} \oint dz \frac{f(z)}{(z-z_0)^2}.
\end{equation}
Equivalently we could just take a derivative with respect to $z_0$ directly. In general
\begin{equation}
    f^{(n)}(z_0) = \frac{n!}{2\pi i} \oint_C dz \frac{f(z)}{(z-z_0)^{n+1}}.
\end{equation}
This tells us that for analytic functions, it's not just the case that the first derivatives exist, but in fact all higher derivatives exist, for the integrand is perfectly well-behaved.

We shall now prove the converse of Cauchy's theorem, that the contour integral over any closed loop vanishing implies analyticity. This is known as Morera's theorem, and we'll do it in a way that improves upon the text.

Suppose that $\oint f(z)=0$ over any closed loop in a region. it follows that for such a loop, we can take two points $z_1,z_2$ on the loop and split it into two paths, such that the integral over each path is equal and opposite:
\begin{equation}
    F(z_2)-F(z_1) = \int_{z_1}^{z_2} dz f(z).
\end{equation}
Now we write it in the following way:
\begin{align*}
    \frac{F(z_2)-F(z_1)}{z_2-z_1} -f(z_1) &= \frac{\int_{z_1}^{z_2} dz f(z) }{z_2-z_1} -f(z_1)\\
        &=\frac{\int_{z_1}^{z_2} dz (f(z)-f(z_1)) }{z_2-z_1}.
\end{align*}
At this point, the book claims that because $f$ is continuous we can replace it by its derivative. But that assumes that the derivative of $f$ exists, which is what we're trying to prove! The right way to say this is that $f(z) - f(z_1) <\epsilon$ as $z\to z_1$, so then
\begin{equation}
    \lim_{z_2\to z_1} \frac{\int_{z_1}^{z_2} dz (f(z)-f(z_1)) }{z_2-z_1} < \epsilon \frac{z_2-z_1}{z_2-z_1} = \epsilon.
\end{equation}
Hence
\begin{equation}
    \lim \frac{F(z_2)-F(z_1)}{z_2-z_1} -f(z_1) =0 \implies F'(z_1) = f(z_1).
\end{equation}
Now we know that $f'(z_1) = F''(z_1)$, and all higher derivatives exist.%revisit this

Consider the function
\begin{equation}
    f(z) = \sum_{n=0}^\infty a_n z^n
\end{equation}
on a circle of radius $R$. If $|f|< M$ (the function is bounded on the circle), then we can prove that
\begin{equation}
    |a_n|< \frac{M}{R^n}.
\end{equation}
For notice that
\begin{equation}
    a_n = \frac{1}{n!} f^{(n)}(z) |_{z=0} =\frac{1}{2\pi} \oint dz \frac{f(z)}{z^{n+1}}.
\end{equation}
Then
\begin{equation}
    |a_n| = \frac{1}{n!} |f^{(n)}(z)|_{z=0} = \frac{1}{2\pi} |\oint dz \frac{f(z)}{z^{n+1}}| < \frac{1}{2\pi} \int_0^{2\pi} d\phi \frac{M}{R^n} = \frac{M}{R^n}.
\end{equation}
In fact, it follows that the only function which is analytic and bounded as $R\to \infty$ is in fact the constant function.

Let us now prove the fundamental theorem of algebra, that a polynomial of order $N$ has $N$ complex roots counting degeneracy. Consider a polynomial $P(z)$, and suppose that $P(z)$ has no root. Then the quantity $1/P(z)$ is perfectly finite and bounded, which means that by our theorem, $1/P(z)$ is actually constant, so it's not a polynomial.

Thus it must have at least one root $z_0$, and we can define $\hat P = \frac{P(z)}{z-z_0}$. This process can terminate no sooner than after $N$ steps, when what's left really is constant.

The real power of analyticity is in the procedure of analytic continuation, however. 
%We don't know what this does on the circle. Except we *do* know. We just haven't realized that we know.
Consider a point $z_0$, where the nearest singularity of a function $f(z)$ lies at $z_1$. The value of the function $f(z')$ at some $z'$ nearby is then
\begin{equation}
    f(z') = \frac{1}{2\pi i} \oint dz \frac{f(z)}{z-z'} = \frac{1}{2\pi i} \oint dz \frac{f(z)}{z-z_0} \sum \paren{\frac{z'-z_0}{z-z_0}}^n = \sum \frac{(z'-z_0)}{2\pi i} \oint dz \frac{f(z)}{(z-z_0)^{n+1}}.
\end{equation}
This is a Taylor series at $z'$, and now
\begin{equation}
    f(z') = \sum_{n=0}^\infty \frac{f^{(n)}(z_0)}{n!} (z'-z_0)^n.
\end{equation}
So in fact the function at a point and its derivatives gave us the function everywhere within a circle up to the closest singularity.